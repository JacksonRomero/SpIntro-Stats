\def\theTopic{Energy Drinks }
\def\dayNum{14}

\begin{center}
{\bf {\large Energy Drinks}}
\end{center}
\vspace{-.1in}

From Red Bull to Monster to -- you name it -- in the last few years
we've seen a large increase in the availability of so called ``Energy
Drinks''. 

{\bf Share and discuss your responses to each of the following
  questions with your group. }
\begin{enumerate}
  \item  Why are energy drinks popular? 
\begin{students}
    \vspace{4cm}    
\end{students}

\begin{key}
  {\it  AWV}
\end{key}

  \item  What claims are made in the advertising of energy drinks?
\begin{students}
     \vspace{5cm}    
\end{students}

\begin{key}
  {\it  AWV }
\end{key}

  \item  How do energy drinks interact with alcohol?
\begin{students}
    \vspace{5cm}    
\end{students}

\begin{key}
  {\it  AWV}
\end{key}

\item An experiment tried to compare the effects of energy drinks with
  and without alcohol on human subjects.  Pharmacology is the study of
  how drugs affect the body, and ``psychopharmacology'' studies
  effects of drugs on the nervous system. An article in {\it Human
    Psycopharmacology} in 2009 reported on an experiment intended to
  tease out some of the effects and to compare an energy drink without
  alcohol to one with alcohol and to a non-energy drink.  The research
  question is:

  {\sf
    Does neuropsychological performance (as measured on the RBANS
    test) change after drinking an energy drink?  After drinking an
    energy drink with alcohol? }

   Higher RBANs scores indicate better memory skills. 

Go to the site:\\
  \url{http://shiny.math.montana.edu/jimrc/IntroStatShinyApps}. 
  Select \fbox{Enter / Describe Data} under \fbox{\sf One of
    Each}. Select \fbox{PreLoaded Data} as the data entry method, and
  select \fbox{REDAvsCntrl} and \fbox{Use These Data} to load today's
  data. 
 
 \begin{fmpage}{.3\linewidth}
 {\footnotesize
 \begin{verbatim}
 treatment RBAN
 REDA	6.84
 REDA	-9.83
 REDA	-0.02
 REDA	-9.12
 REDA	-10.07
 REDA	-19.34
 REDA	3.97
 REDA	-16.37
 REDA	-21.02
 Control	6.33
 Control	1.65
 Control	-3.58
 Control	3.3
 Control	-6.6
 Control	3.29
 Control	1.8
 Control	1.8
 Control	2.98
 \end{verbatim}
 }
 \end{fmpage}
 \hfill
 \begin{minipage}{.6\linewidth}

   Examine the boxplots and dotplots. Describe any differences in the
   response (Change in RBANS) you see  between Red+A and Control groups. \\

   Center
\begin{students}
    \vspace{1.4cm}    
\end{students}

\begin{key}
  {\it Mean of RED+A is clearly  lower (-8.3), Control  (1.2)\\
  Similarly, median for RED+A is -9.8; median for Control is 1.8 }
\end{key}

   Spread  (SD and IQR)
\begin{students}
    \vspace{1.4cm}    
\end{students}

\begin{key}
  {\it      Control has smaller spread (SD = 3.9, IQR = 1.64)
     compared to  RED+A (SD = 10.0, IQR = 10) 
}
\end{key}


   Shape
\begin{students}
    \vspace{1.4cm}    
\end{students}

\begin{key}
  {\it            RED+A seems most symmetric (boxplot), Cotrol
         has some outliers.
}
\end{key}
\end{minipage}


The researchers used a computer randomization to assign the 
subjects into the groups. We'll shuffle cards instead. 

\item  Take 18 index cards and write the numbers 1 through 18 in a top
  corner, and the score of each individual in the middle starting with
  6.84 for card 1, on to -21.02 for card 9, then continue with the
  second row. Line them up in the two rows  like this data table: 

  \begin{tabular}{l|rrrrrrrrrr} \hline
RED+A & 6.84&-9.83&-0.02&-9.12&-10.07&-19.34&3.97&-16.37&-21.02&\\ \hline
Control&6.33&1.65&-3.58&3.30&-6.60&3.29&1.80&1.80&2.98&\\ \hline
\end{tabular}\\
Compute the two means and take their difference: $\xb_1 - \xb_2$.
\vspace{1in}


\item We want to test the hypothesis that the means are equal:\\
  $H_0: \mu_1 = \mu_2$  (no difference in mean 'change in RBANS score'
  between REDA and control groups.)  versus:\\
  $H_a:  \mu_1 \neq \mu_2$\\
\end{enumerate}
Consider this important question:\\

{\bf \sf
If the treatments have no effect on RBANS scores, then where do the
observed differences in distributions and in means come from?
}\vspace*{.2cm}


\begin{enumerate}
  \setcounter{enumi}{5}
\item 
   Discuss this within your group and write down your answer.
  Don't say that it has anything to do with the drink they were given
  because we are assuming the drinks are all having the same effect. 
  (Give this about 2 minutes of serious discussion, then go on.
   If you get stuck here, we  won't have time to finish the activity.)
\begin{students}
    \vspace{7cm}    
\end{students}

\begin{key}
  {\it   Random variation}
\end{key}

\item  Turn the index cards over and slide them around or shuffle by
  some other method, until you think they are thoroughly mixed up.
  Line up the shuffled cards (turned face up again) in 2 rows of 9 and
  compute the mean of each row. 
  \vspace{1cm} \\

  Does the difference in the new means agree well with the original
  difference?   If not,  how much has it changed? 
\begin{students}
    \vspace{3cm}    
\end{students}

\begin{key}
  {\it  They should be similar but not exactly the same.}
\end{key}


\item  Suppose the first persons' change in RBANs was going to be
  6.824 no matter which drink she was given, that the second would
  always be -9.83, and so on to the last person's score of 2.98.  If
  we re-shuffle the people and deal them into two groups of 9 again
  and label then  RED+A and Control, why do the means change? (You
  are describing a model of how the data are generated) 
\begin{students}
    \vspace{5cm}    
\end{students}

\begin{key}
  {\it     The scores within each group change so the means change.  The person
  from RED+A no longer has to be in RED+A, etc.  Who was in which
  treatment group is randomized. 
}
\end{key}


 \item  Go back to the applet and select \fbox{Test} under \fbox{One of Each}.
   \begin{enumerate}
   \item 
   Do the means and SD's in the summary table match what we had
   earlier?  Did they subtract in the same order as we did?
\begin{students}
    \vspace{1cm}    
\end{students}

\begin{key}
  {\it  They should! If it's reversed, swap the order in the data.}
\end{key}

    \item  What are the means for control and RED+A in the
      reshuffled version?  The difference?
\begin{students}
    \vspace{1cm}    
\end{students}

\begin{key}
  {\it  AWV, but difference should be closer to 0.}
\end{key}



 \item Explain how our shuffling the cards is like what the computer does to
   the data.
\begin{students}
    \vspace{3cm}    
\end{students}

\begin{key}
  {\it  The computer just randomized which group each score came from
   similar to shuffling the scores and replacing them into different
   groups.  }
\end{key}

 \item Click \fbox{\sf 1000}   three times.  Where is the plot
   centered?   Why is it centered there?
\begin{students}
    \vspace{2cm}    
\end{students}

\begin{key}
  {\it     Centered close to 0 because the null is that treatment has no
   effect (i.e. the means are the same). }
\end{key}
%\vspace{3cm}


 \item  Below the plot, keep  \fbox{\sf more extreme}  and  enter the
   {\bf observed difference in means from the original data} in the
   last box.  Click \fbox{Go}. What proportion of the samples are this
   extreme?   
\begin{students}
    \vspace{2cm}    
\end{students}

\begin{key}
  {\it 0.008 in my sample. }
\end{key}

   \end{enumerate}

 \item  There are other reasons that one person might show more change
   in RBANS than another person.  Write down at least one. (Again,
   don't get stuck here.) 
\begin{students}
    \vspace{2cm}    
\end{students}

\begin{key}
  {\it   Genetic differences, difference in education levels, difference
     in amount of sleep the previous night, etc. 
}
\end{key}
\vspace{6cm}

 \item  Lurking variables were discussed on Activity 10.  When we
   randomly assign treatments, how should the groups compare on any
   lurking variable? 
\begin{students}
    \vspace{3cm}    
\end{students}

\begin{key}
  {\it  The should be approximately equivalent in the long run (or on average). }
\end{key}
   
 \item  Are you willing to conclude that the differences we see
   between the two groups are caused by REDA?  Explain
   your reasoning. 
\begin{students}
    \vspace{4cm}    
\end{students}

\begin{key}
  {\it  
   Most will say probably no because of the lurking variables or
   sample size (?) but we {\bf can} since we have random assignment.
 }
\end{key}
%    \item We will also estimate the difference in the true
%      mean RBANS change (REDA versus control) using StatKey.  Select
%      the {\sf  CI for Difference in Means} option.
%      \begin{enumerate}
%      \item Click \fbox{\sf Edit Data} and delete all of their data.
%        Insert the RBANS data from D2L. Add commas between the label
%        and RBANS value in each line. Click \fbox{\sf OK} and check the
%        sample statistics right under {\sf Original Sample}. Do they
%        agree with what we had above?
% \begin{students}
%     \vspace{2cm}    
% \end{students}

% \begin{key}
%   {\it  If not, data were entered wrong.  }
% \end{key}
%      \item Click \fbox{\sf Generate 1 Sample}, then generate 1000
%        more.  The app is creating a 
%        bootstrap resample from the 9 REDA values and a bootstrap
%        resample from the 9 Control values, finding the mean of each,
%        and subtracting to get a difference in means which shows up in
%        the plot.  Click again until you are positive you know what it
%        is doing. For your last single sample, write the two bootstrap
%        means and their difference.
% \begin{students}
%     \vspace{2cm}    
% \end{students}

% \begin{key}
%   {\it  I got 1.09 for Control, -7.97 for REDA with difference = -9.06.  }
% \end{key}
%     \item Obtain a 95\% bootstrap percentile interval by clicking {\sf
%         Two-Tail} and write it here.
% \begin{students}
%     \vspace{2cm}    
% \end{students}

% \begin{key}
%   {\it (-15.99, -2.70)  }
% \end{key}
    \item Write your interpretation of this interval.
\begin{students}
    \vspace{1cm}    
\end{students}

\begin{key}
  {\it  We are 95\% confident that the true mean ``change in RBANS score''
  is  2.7 to 15.99 points lower when people drink REDA than when they
   drink the control beverage. Be sure to indicate which one is higher
 (or lower). } 
\end{key}
\item Build a confidence interval using ``estimate $\pm t^*$SE'' where the
  estimate is the observed difference in means, $t^* = 2.12$, and
  using the st.dev. from the plot as our SE.  Does it contain zero?
 \begin{students}
    \vspace{2cm}    
\end{students}

\begin{key}
  {\it (-16.65, -2.45)  No.}
\end{key}


\item Write up the results of the hypothesis test as a report using
  the five elements from Activity 12.  Be sure to refer to the response
  variable as ``change in RBANS'', not just RBANS score. \\ \ \\

\begin{students}
 \newpage 
\end{students}
 \end{enumerate}


{\sf\bf Take Home Messages:}
\begin{itemize}
  \item 
  If there is no treatment effect, then differences in distribution
  are just due to the random assignment of treatments.  This
  corresponds to a ``null hypothesis'' of no difference between
  treatment groups.
\item  By randomly applying treatments, we are creating groups that
  should be very similar because differences between groups (age,
  reaction to alcohol, memory) are “evened out” by the random group
  allocation. If we see a difference between groups, then we doubt the
  null hypothesis that treatments don't matter.  Any difference
  between groups is caused by the treatment applied.  Random
  assignment is a very powerful tool.  When reading a study, it's one
  of the key points to look for. 
 \item 
  Use the remaining space for any questions or your own summary of the
  lesson. \vfill

\end{itemize}
{\sf\bf Reference}

 Curry K, Stasio MJ.  (2009). The effects of energy drinks alone and with
 alcohol on neuropsychological functioning. {\it Human  Psychopharmacology}.
{\bf 24}(6):473-81. doi: 10.1002/hup.1045. 


