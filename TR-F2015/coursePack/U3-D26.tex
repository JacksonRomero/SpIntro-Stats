\def\theTopic{Effects of Concussion }
\def\dayNum{26 }

\begin{center}
{\bf {\large Concussions in the News}}\\
\end{center}

\begin{itemize}
\item 
Chris Borland, age 24,  retired from the 49ers after one year of
NFL play, because he feared his brain might be permanently injured by
concussions.
\item In 2013 the NFL agreed to pay \$765 million to fund exams,
  research and for compensation of former players.
\item High school coaches are now advised to make their
  players sit out if they sustain a blow to the head. 
\item The Congressional Brain Injury Task Force is looking into 
  the prevalence of brain trauma in troops deployed in recent wars and
  at how VA hospitals are treating the condition.
\end{itemize}

\begin{center}
  {\large\bf Applying What We've Learned}
\end{center}


 Some have argued that blows to the head -- for instance on the
    high school football field -- are getting too much attention.
    They point out that of many kids taking similar hits, only a few
    show lasting decrease in cognitive abilities, and suggest that the
    effects are as  much due to genetics as to the blow to the head.
    \begin{enumerate}
    \item Your group is asked to design a study to
      determine how large a risk the ``hits'' taken by high school
      athletes are to their future cognitive abilities. Write down
      a plan for your study.  Include:
      \begin{itemize}
      \item Who will your subjects be?  How will you find them? How
        many will you need? 
      \item What will you measure? (Brain scans like MRI?  Cognitive
        tests of memory and reasoning?  Some measure of emotional
        states like anger?) Include the timing of measurements.
      \item Will your study be observational or an experiment?
      \item Is your study ethical in its treatment of subjects?
      \end{itemize}
      %%  15 minutes?
   \newpage
      \vspace*{2.5in}

    \item Trade papers with another group and read their study over
      carefully. Address the bullet points above  (gently --
      remember someone is doing this to your proposal as well).
      Provide suggestions as to what might be improved.
      \begin{itemize}
      \item Subjects?
      \item Measurements?
      \item Observational? Experiment?
      \item Ethical issues?
      \end{itemize}
      \vspace{5in}
       %% 10 minutes?  

  
 % \item Why is it hard to figure out how large a role genetics play
 %      in the effects we observe?  \vspace{2in}

%   \item   Talavage, Thomas M., et al. ``Functionally-detected
%     cognitive impairment in high school football players without
%  clinically-diagnosed concussion.'' {\em Journal of Neurotrauma} {\bf
%    31}.4 (2014): 327-338.\\ 
% \url{http://online.liebertpub.com/doi/abs/10.1089/neu.2010.1512}
%   Obs study followed  11 HS players observed 
%   \begin{enumerate}
%   \item  no concussion \& no changes
%   \item  concussion \& changes
%   \item  no concussion \& changes
%   \end{enumerate}

%     What are the advantages of this study
%     design?  Disadvantages?  Does the design allow us to make causal
%     inferences?  To high school students?
    
%   \item 
% Brain Connect. 2015 Apr;5(2):91-101. doi: 10.1089/brain.2014.0279. Epub 2014 Oct 21.
% ``Alteration of default mode network in high school football athletes due to repetitive subconcussive mild traumatic brain injury: a resting-state functional magnetic resonance imaging study.''
% Abbas K1, Shenk TE, Poole VN, Breedlove EL, Leverenz LJ, Nauman EA,
% Talavage TM, Robinson ME.
% \url{http://www.ncbi.nlm.nih.gov/pubmed/25242171}

% \begin{quotation} 
% Abstract\\
% Long-term neurological damage as a result of head trauma while playing
% sports is a major concern for football athletes today. Repetitive
% concussions have been linked to many neurological disorders. Recently,
% it has been reported that repetitive subconcussive events can be a
% significant source of accrued damage. Since football athletes can
% experience hundreds of subconcussive hits during a single season, it
% is of utmost importance to understand their effect on brain health in
% the short and long term. In this study, resting-state functional
% magnetic resonance imaging (rs-fMRI) was used to study changes in the
% default mode network (DMN) after repetitive subconcussive mild
% traumatic brain injury. Twenty-two high school American football
% athletes, clinically asymptomatic, were scanned using the rs-fMRI for
% a single season. Baseline scans were acquired before the start of the
% season, and follow-up scans were obtained during and after the season
% to track the potential changes in the DMN as a result of experienced
% trauma. Ten noncollision-sport athletes were scanned over two sessions
% as controls. Overall, football athletes had significantly different
% functional connectivity measures than controls for most of the
% year. The presence of this deviation of football athletes from their
% healthy peers even before the start of the season suggests a
% neurological change that has accumulated over the years of playing the
% sport. Football athletes also demonstrate short-term changes relative
% to their own baseline at the start of the season. Football athletes
% exhibited hyperconnectivity in the DMN compared to controls for most
% of the sessions, which indicates that, despite the absence of symptoms
% typically associated with concussion, the repetitive trauma accrued
% produced long-term brain changes compared to their healthy peers.
%  \end{quotation}


\item You should have read the abstract of this article for today:

Petraglia AL, Plog BA, Dayawansa S, Chen M, Dashnaw ML, Czerniecka K,
Walker CT, Viterise T, Hyrien O, Iliff JJ, Deane R, Nedergaard M,
Huang JH. (2014). ``The spectrum of neurobehavioral consequences after repetitive mild traumatic brain injury: a novel mouse model of chronic traumatic encephalopathy.''
{\bf J Neurotrauma} Jul 1; {\bf 31}(13):1211-24. 
\url{http://www.ncbi.nlm.nih.gov/pubmed/24766454}

\begin{enumerate}
\item  What are the advantages of this study design?
\begin{students}
 \vfill
\end{students}

\begin{key}
  Replication, control, random assignment, very similar subjects.
\end{key}


\item Disadvantages?
\begin{students}
 \vfill
\end{students}

\begin{key}
 Does not extend to humans directly -- only if we assume some linkage.
\end{key}



\item Does the design allow us to make causal
    inferences?
\begin{students}
 \vspace{1cm}
\end{students}

\begin{key}
Yes
\end{key}

\item Inferences to high school students?

\begin{students}
\vspace*{.6cm}
\newpage
\end{students}

\begin{key}  No
\end{key}



\end{enumerate}

  \item And the abstract for this article:

Lin, Ramadan, Stern, Box, Nowinski, Ross, Mountford. (2015).
``Changes in the neurochemistry of athletes with repetitive brain trauma: preliminary results using localized correlated spectroscopy.''
{\bf Alzheimers Research \& Therapy}. 2015 Mar 15;7(1):13
\url{http://www.ncbi.nlm.nih.gov/pubmed/25780390}

\begin{enumerate}
\item  What are the advantages of this study design?
\begin{students}
 \vfill
\end{students}

\begin{key}
 Ethically-- they didn't hurt anyone.
\end{key}

\item Disadvantages?
\begin{students}
 \vfill
\end{students}

\begin{key}
No random assignment.  Small samples.
\end{key}

\item Does the design allow us to make causal
    inferences?
\begin{students}
 \vspace{.6cm}
\end{students}

\begin{key} No
\end{key}



\item Inferences to high school students?
\begin{students}
\vspace*{.6cm}
\newpage
\end{students}

\begin{key} No
\end{key}

\end{enumerate}

\item And this quote:
{\footnotesize
  \begin{quotation}
Mielke and her colleagues did PET scans on 589 people who are
participating in a long-term study of aging and memory. That's a lot
of people for a brain imaging study, which makes it more likely that
the findings are accurate. 
  \end{quotation}
}
\url{http://www.npr.org/blogs/health/2013/12/27/257552665/concussions-may-increase-}
\url{alzheimers-risk-but-only-for-some}

Explain the last sentence.  In what sense does a large sample increase accuracy?

\begin{students}
 \vspace{4cm}
\end{students}

\begin{key}
   Larger sample sizes give smaller variance estimates to $\xb$ and
   $|phat$.  That tends to shrink p--values and increase power, so we
   are less likely to make Type I or Type II errors. Larger samples do
   not decrease bias, so we'd still like to have random samples from
   the population.
\end{key}


\end{enumerate}


\begin{center}
  {\large\bf Take Home Message}
\end{center}

\begin{itemize}
\item  In reading a research article, carefully read what they say
  about selection of subjects (or units of study) so we know how far
  we can extend our inference.
\item Similarly, evaluate the assignment of any treatments (at random,
  we hope) so that we know if causal inference is appropriate.
\item In designing a study, we'd like to get a large sample, but
  expense might prevent that.
\item You might see names of tests we've not covered, but the idea of
  a p-value is the same for any test, and the null hypothesis is
  generally ``No change'', or ``No effect''.
\item What's not clear?  Write your questions here.
% P-values may be too small and confidence intervals may be too narrow
% if there is other variation present besides the ``sampling
% distribution''.  The methods you've learned in this class only cover
% the variation we get from randomly sampling subjects and randomly
% assigning treatments.
\end{itemize}


