
\def\theTopic{Confidence }
\def\dayNum{7}

\begin{center}
\vspace*{-.2in}
{\bf {\large What Does ``Confidence'' Mean?}}\\
\end{center}

Mark Twain said:
\begin{quotation}
All you need in this life is ignorance and confidence, and then
success is sure.   
\end{quotation}

 from quarterback Joe Namath:
\begin{quotation}
When you have confidence, you can have a lot of fun. And when you have fun, you can do amazing things.  
\end{quotation}

and from scientist Marie Curie:
\begin{quotation}
  Life is not easy for any of us. But what of that? We must have
  perseverance and above all confidence in ourselves. We must believe
  that we are gifted for something and that this thing must be
  attained. 
\end{quotation}

The above quotes (from brainyquote.com) refer to  ``self confidence''
which is certainly important in any endeavor.
In statistics, the word ``confidence'' is best summarized as {\bf
  faith in the process} by which an estimate (in our case, an interval
estimate) was created.  A confidence interval carries information
about the {\bf location} of the parameter of interest, and tells us a lot
about the {\bf precision} of the estimate through the interval
length. 


In the news, interval estimates are often reported as a point value
and a {\bf margin of error}. 

\begin{quotation}
  71\% of Democrats and independents who lean to the Democratic Party
  say the Earth is warming due to human activity, compared with 27\%
  among their Republican counterparts (a difference of 44 percentage
  points). This report shows that these differences hold even when
  taking into account the differing characteristics of Democrats and
  Republicans, such as their different age and racial profiles. 
\end{quotation}

  If you read the ``fine print'' from the  Pew Research Center which
  conducted the poll,
\url{http://www.pewinternet.org/2015/07/01/appendix-a-about-the-general-public-survey-2/}
you find that the margin of error for a 95\% confidence interval is
listed as $\pm$ 5.1\% for Republican/lean Republican and $\pm 4.5$\%
for Democrat/lean Democrat proportions.

\begin{center}
  {\Large \bf Plus or Minus Confidence Intervals}
\end{center}

In the web app used in previous activities, we clicked on a confidence
level and the web app colored in the right number of dots as red to
put our selected percentage of sampled proportions in the center
(these stayed blue) and split the remainder into the two tails,
turning these more extreme points red.  We call this a ``percentile''
method because, for example, a 90\% CI has lower endpoint of the 5th
percentile and upper endpoint of the 95th percentile.

Another common way of building a 95\% confidence interval is to take
the estimated value and add and subtract twice the standard error of
the statistic.  A 95\% confidence interval for $p$ is then
 $$ \phat \pm 2 SE(\phat)$$
where $SE(\phat)$ is a number coming from the plot on the web app.

Margin of error is then the amount we add and subtract.  In this case,
it is twice $SE(\phat)$.  (Note: the parentheses do not mean
multiplication, say of SE times $\phat$. They indicate that $SE$ is a
function of $\phat$, in the same way we use $\log(x)$ or $\sin(\theta)$.)

\begin{enumerate}
\item Go back to the rat data from Activity 6 where 23 rats opened the
  cage and 7 did not.  Reenter the data in the \fbox{One Categ} part
  of the web app, and select \fbox{Estimate}. 
  \begin{enumerate}
  \item Generate 5000 to 10,000 resamples and click 95\%. Record the
    interval here:
\begin{students}
\vspace{.8cm}
\end{students}

\begin{key}
  {\em (0.60, 0.90)}
\end{key}
\item Now write down the SE shown near the top right corner of the
  plot.  (We will not use the mean of the plotted values).
\begin{students}
\vspace{.8cm}
\end{students}

\begin{key}
  {\em 0.077}
\end{key}
\item Add and subtract $2SE$ from the original proportion given in the
  box at left ( {\bf Do not} use the mean from the plot.) and write it
  in interval notation.

\begin{students}
\vspace{.8cm}
\end{students}

\begin{key}
  {$ 0.77 \pm 2\times 0.077 =  (0.63, 0.91)$}
\end{key}
\item Compare the two intervals.  Is one wider? Is there a shift?

\begin{students}
\vspace{.8cm}
\end{students}

\begin{key}
  {\em The percentile CI is shifted slightly to the left and is
    slightly wider.}
\end{key}

  \end{enumerate}
\end{enumerate}

\begin{center}
  {\Large\bf Meaning of ``Confidence''}
\end{center}

To understand the meaning of the term ``confidence'', you have to step
back from the data at hand and look at the process we use to create
the interval.
\begin{itemize}
  \item Select a random sample from a population, measure each unit,
    and compute a  statistic like $\phat$ from it.
  \item Resample based on the statistic to create the interval.
  \end{itemize}

To check to see how well the techniques work, we have to take a
 special case where we actually know the true parameter value.
 Obviously, if we know the value, we don't need to estimate it, but we
 have another purpose in mind: we will use the true value to generate
 many samples, then use each sample to estimate the parameter, and
 finally, we can check to see how well the confidence interval
 procedure worked by looking at the proportion of intervals which
 succeed in capturing the parameter value we started with.

 Again go to \url{http://shiny.math.montana.edu/jimrc/IntroStatShinyApps/}
 and select ``Confidence Interval Demo'' from the ``One Categ'' menu.
 
 The first slider on this page allows us to set the sample size --
 like the number of units or subjects in the experiment.  Let's start with
 \fbox{40}.\\
 The second slider sets the true proportion of successes for each
 trial or spin (one trial).  Let's set that at \fbox{0.75} or 75\%
 which is close to the observed $\phat$ of the rat study.\\
 You can then choose the number of times to repeat the process (gather
 new data and build a confidence interval [CI]: 10, 100, 1000 or 10K
 times) and the level of confidence you want (80, 90, 95, or 99\%).\\
 We'll start wih \fbox{100} simulations of a \fbox{90}\% CI.

  The upper plot shows 100  $\phat$'s -- one from each of the 100 simulations.
  \\
  The second plot shows the interval estimate we get from each
  $\phat$.  These  are stacked up to put smallest estimates on the
  bottom, largest on top. The vertical axis has no real meaning. 

  \begin{enumerate}
    \setcounter{enumi}{1}
  \item   Click on a point in the first plot to see its corresponding CI in
  the second plot.  Especially try the largest and smallest points.
  Which intervals do they create (in terms of left or right position)?
\begin{students}
  \vspace{1cm}
\end{students}
\begin{key}
  {\it lowest and highest CI's, resp.}
\end{key}
\item There is a light gray vertical line in the center of the lower
  plot. What is the value (on the $x$ axis) for this plot and why is
  it marked?
\begin{students}
  \vspace{1cm}
\end{students}
\begin{key}
  {\it It is the true parameter value: 0.75 if you followed the directions.}
\end{key}
\item What color are the intervals which do not cross the vertical
  line? \\How many are there?
\begin{students}
  \vspace{1cm}
\end{students}

\begin{key}
  {\it red, AWV about 10}
\end{key}

\item What color are the intervals which cross over the vertical
  line? \\How many are there?
\begin{students}
  \vspace{1cm}
\end{students}

\begin{key}
  {\it green, AWV about 90}
\end{key}

\item Change the confidence level to \fbox{95}\%. Does the upper plot
  change?  Does the lower plot?  Describe any changes.
\begin{students}
  \vspace{1cm}
\end{students}
\begin{key}
  {\it Upper plot should not change. Each interval in the lower plot
    gets longer, so some that were red may turn green now.}
\end{key}


\item If you want an interval which is stronger for confidence
  (has a higher level), what will happen to its width?
\begin{students}
  \vspace{.6cm}
\end{students}
\begin{key}
  {\it it must be wider}
\end{key}

  \item Go up to 1000 or more intervals, try each confidence level in
    turn and record the coverage rate   (under plot 2) for each.\\
    \begin{tabular}{|r|r|r|r|} \hline
      {\Large 80} &  {\Large 90} &  {\Large 95} &  {\Large 99}\\ \hline
   {\Large  \phantom{9000} } & {\Large \phantom{9000}  } &  {\Large
     \phantom{9000} } &  {\Large  \phantom{9000} } \\
       & & & \\ \hline
    \end{tabular}

  \item Now back to the Pew study we started with. Of the 2002 people
    they contacted, 737 were classified as Republican (or Independents
    voting Rep) voters and 959 as Democrats (or Indep leaning Dem).
    \begin{enumerate}
    \item What integer number is closest to 27\% of the Republicans?
      Enter that value into the \fbox{One Categ} ``Enter Data'' boxes
      as successes (relabel success as you see fit)  and the balance
      of those 737 in the bottom box. Check that the proportion on the
      summary page is  close to 0.27. 
      \begin{enumerate}
      \item What is your proportion of Republicans who think global
        warming is caused by human activity?
\begin{students}
\vspace{.8cm}
\end{students}

\begin{key}
  {\em 199 ``successes'', 538 ``Failures'', $\phat = $0.27}
\end{key}
      \item Go to the ``Estimate'' page and run several 1000
        samples. What is the standard error?
\begin{students}
\vspace{.8cm}
\end{students}

\begin{key}
  {\em 0.016}
\end{key}
      \item Find the ``margin of error'' for a 95\% Confidence
        interval and write down the interval.

\begin{students}
\vspace{.8cm}
\end{students}

\begin{key}
  {\em ME = 0.032, 95\% CI: 0.27$\pm 0.032 = (0.38, 0.302)$}
\end{key}
  \item Compare that interval to one you get by clicking the
    appropriate button in the app.

\begin{students}
\vspace{.8cm}
\end{students}

\begin{key}
  {\em almost identical: ( 0.237 , 0.303 )}
\end{key}
      \end{enumerate}
    \item Repeat for the Democrats:
      \begin{enumerate}
      \item Numbers of ``successes'' and ``failures''.
\begin{students}
\vspace{.8cm}
\end{students}

\begin{key}
  {\em 700, 259 }
\end{key}
      \item Margin of error and 95\% CI related to it.
\begin{students}
\vspace{.8cm}
\end{students}

\begin{key}
  {\em 0.028, (0.702, 0.758)}
\end{key}
      \item Percentile interval and comparison.
\begin{students}
\vspace{.8cm}
\end{students}

\begin{key}
  {\em (0.701, 0.758), again very close. }
\end{key}
      \end{enumerate}
    \item Explain what we mean by ``confidence'' in these intervals we
      created.
\begin{students}
\vspace{4.8cm}
\end{students}

\begin{key}
  {\em We are 95\% confident that the true }
\end{key}

\item What can we say about the proportions of Republicans and the
  proportion Democrats on this issue? Is it conceivable that the
  overall proportion is the same?  Explain.

\begin{students}
\vspace{2.8cm}
\end{students}

\begin{key}
  {\em The intervals do not come close to overlapping, so we have to
    think that there is a strong difference of opinion between these
    two groups. I am ``quite confident'' of that.}
\end{key}
    \end{enumerate}
  
  \end{enumerate}


\begin{center}
  {\large \bf Take Home Message} 
\end{center}

\begin{itemize}
\item Interval estimates are better than point estimates.
\item Our confidence in a particular interval is actually in the
  process used to create the interval.  We know that using this
  process over and over again (go out and collect a new random sample
  for each time) gives intervals which will usually
  cover the true value.\\
   We cannot know if a particular interval covered or not, so you have
   to be tolerant of some uncertainty.
 \item 
  Use the remaining space for any questions or your own summary of the
  lesson. 
\end{itemize}




