\def\theTopic{Reading 3}

\begin{center}
{\bf {\large Helper Hinderer}}
\end{center}

Researchers at Yale University were interested in how soon in human
development children become aware of (and start to favor) activities
that help rather than hinder others.


Title: ``Social evaluation by preverbal infants''

Authors: J. Kiley Hamlin, Karen Wynn \& Paul Bloom

Journal: {\it Nature} 450, 557-559 (22 November 2007) 

Abstract

\begin{quotation}
  The capacity to evaluate other people is essential for navigating the
social world. Humans must be able to assess the actions and intentions
of the people around them, and make accurate decisions about who is
friend and who is foe, who is an appropriate social partner and who is
not. Indeed, all social animals benefit from the capacity to identify
individuals  that may help them, and to distinguish these
individuals from others that may harm them. Human adults evaluate
people rapidly and automatically on the basis of both behaviour and
physical features, but the origins and
development of this capacity are not well understood. Here we show
that 6- and 10-month-old infants take into account an individual's
actions towards others in evaluating that individual as appealing or
aversive: infants prefer an individual who helps another to one who
hinders another, prefer a helping individual to a neutral individual,
and prefer a neutral individual to a hindering individual. These
findings constitute evidence that preverbal infants assess individuals
on the basis of their behaviour towards others. This capacity may
serve as the foundation for moral thought and action, and its early
developmental emergence supports the view that social evaluation is a
biological adaptation. 



The following were randomized across subjects:
(1) colour/shape of helper and hinderer; (2) order of helping and
hindering events; (3) order of choice and looking time
measures; and (4) positions of helper and hinderer. 
\end{quotation}

\newpage

 \begin{center}
   {\large\bf Important Points}
 \end{center}
 \begin{itemize}
 \item What is the research question?\vspace{1in}
 \item What response was recorded? What type of variable is the
   resposne? \vspace{1in}
 \item How was randomness utilized?\vspace{1in}

 \end{itemize}




