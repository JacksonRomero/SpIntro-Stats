\def\theTopic{Concussion Articles }

\begin{center}
{\bf {\large Concussions in the News}}\\
\end{center}


\begin{enumerate}
\item   Read the abstract from this  article.

Petraglia AL, Plog BA, Dayawansa S, Chen M, Dashnaw ML, Czerniecka K,
Walker CT, Viterise T, Hyrien O, Iliff JJ, Deane R, Nedergaard M,
Huang JH. (2014). ``The spectrum of neurobehavioral consequences after repetitive mild traumatic brain injury: a novel mouse model of chronic traumatic encephalopathy.''
{\bf J Neurotrauma} Jul 1; {\bf 31}(13):1211-24. 
\url{http://www.ncbi.nlm.nih.gov/pubmed/24766454}
{\footnotesize
\begin{quotation}
Abstract\\
There has been an increased focus on the neurological consequences of
repetitive mild traumatic brain injury (TBI), particularly
neurodegenerative syndromes, such as chronic traumatic encephalopathy
(CTE); however, no animal model exists that captures the behavioral
spectrum of this phenomenon. We sought to develop an animal model of
CTE. Our novel model is a modification and fusion of two of the most
popular models of TBI and allows for controlled closed-head impacts to
unanesthetized mice. Two-hundred and eighty 12-week-old mice were
divided into control, single mild TBI (mTBI), and repetitive mTBI
groups. Repetitive mTBI mice received six concussive impacts daily for
7 days. Behavior was assessed at various time points. Neurological
Severity Score (NSS) was computed and vestibulomotor function tested
with the wire grip test (WGT). Cognitive function was assessed with
the Morris water maze (MWM), anxiety/risk-taking behavior with the
elevated plus maze, and depression-like behavior with the forced
swim/tail suspension tests. Sleep
electroencephalogram/electromyography studies were performed at 1
month. NSS was elevated, compared to controls, in both TBI groups and
improved over time. Repetitive mTBI mice demonstrated transient
vestibulomotor deficits on WGT. Repetitive mTBI mice also demonstrated
deficits in MWM testing. Both mTBI groups demonstrated increased
anxiety at 2 weeks, but repetitive mTBI mice developed increased
risk-taking behaviors at 1 month that persist at 6 months. Repetitive
mTBI mice exhibit depression-like behavior at 1 month. Both groups
demonstrate sleep disturbances. We describe the neurological consequenses
of repetitive mTBI in a novel mouse model, which resemble several of
the neuropsychiatric behaviors observed clinically in patients
sustaining repetitive mild head injury.
\end{quotation}
}
 Be prepared to answer questions about 
\begin{enumerate}
\item  The questions researchers wanted to answer.
\begin{students}
 \vspace{1in}
\end{students}

\begin{key}
  What happens to mice after repeated brain injury?
\end{key}


\item Who/what were the subjects?
\begin{students}
 \vspace{1cm}
\end{students}

\begin{key}
 mice
\end{key}



\item Were treatments applied at random?
\begin{students}
\newpage
\end{students}

\begin{key}
Yes
\end{key}

\end{enumerate}

  \item And read this abstract:

Lin, Ramadan, Stern, Box, Nowinski, Ross, Mountford. (2015).
``Changes in the neurochemistry of athletes with repetitive brain trauma: preliminary results using localized correlated spectroscopy.''
{\bf Alzheimers Research \& Therapy}. 2015 Mar 15;7(1):13
\url{http://www.ncbi.nlm.nih.gov/pubmed/25780390}
{\footnotesize
\begin{quotation}
  Abstract\\
INTRODUCTION:\\
The goal was to identify which neurochemicals differ in professional
athletes with repetitive brain trauma (RBT) when compared to healthy
controls using a relatively new technology, in vivo Localized
COrrelated SpectroscopY (L-COSY).\\ 
METHODS:\\
To achieve this, L-COSY was used to examine five former professional
male athletes with 11 to 28 years of exposure to contact sports. Each
athlete who had had multiple symptomatic concussions and repetitive
sub concussive trauma during their career was assessed by an
experienced neuropsychologist. All athletes had clinical symptoms
including headaches, memory loss, confusion, impaired judgment,
impulse control problems, aggression, and depression. Five healthy
men, age and weight matched to the athlete cohort and with no history
of brain trauma, were recruited as controls. Data were collected from
the posterior cingulate gyrus using a 3 T clinical magnetic resonance
scanner equipped with a 32 channel head  coil.\\
RESULTS:\\
The variation of the method was calculated by repeated examination of
a healthy control and phantom and found to be 10\% and 5\%,
respectively, or less. The L-COSY measured large and statistically
significant differences (P $<=$0.05), between healthy controls and
those athletes with RBT. Men with RBT showed higher levels of
glutamine/glutamate (31\%), choline (65\%), fucosylated molecules
(60\%) and phenylalanine (46\%). The results were evaluated and the
sample size of five found to achieve a significance level P = 0.05.  Differences in N-acetyl aspartate and myo-inositol
between RBT and controls were small and were not statistically
significance.\\ 
CONCLUSIONS:\\
A study of a small cohort of professional athletes, with a history of
RBT and symptoms of chronic traumatic encephalopathy when compared
with healthy controls using 2D L-COSY, showed elevations in brain
glutamate/glutamine and choline as recorded previously for early
traumatic brain injury. For the first time increases in phenylalanine
and fucose are recorded in the brains of athletes with RBT. Larger
studies utilizing the L-COSY method may offer an in-life method of
diagnosis and personalized approach for monitoring the acute effects
of mild traumatic brain injury and the chronic effects of RBT. 
\end{quotation}
}
 Be prepared to answer questions about 
\begin{enumerate}
\item  The questions researchers wanted to answer.
\begin{students}
 \vspace{1in}
\end{students}

\begin{key}
  Do brain chemicals differ in athletes exposed to brain trauma?
\end{key}


\item Who/what were the subjects?
\begin{students}
 \vspace{1cm}
\end{students}

\begin{key}
 5 athletes and 5 non-athlete men.
\end{key}



\item Were treatments applied at random?
\begin{students}
 \vspace{1cm}
\end{students}

\begin{key}
No
\end{key}

\end{enumerate}


\item Here's a quote from a news report in 2013:
{\footnotesize
  \begin{quotation}
    ``We need to figure out what's making some people more vulnerable
    than others,'' says Michelle Mielke, an Alzheimer's researcher at
    the Mayo Clinic who led the study. It was published online
    Thursday in the journal Neurology. 

``Just because you have a head trauma doesn't mean you're going to
develop memory problems or significant amyloid levels," Mielke told
Shots. And it doesn't mean you're going to get Alzheimer's. "But it
does suggest to us that there's a mechanism with head trauma that does
increase your risk.''

Mielke and her colleagues did PET scans on 589 people who are
participating in a long-term study of aging and memory. That's a lot
of people for a brain imaging study, which makes it more likely that
the findings are accurate. 
  \end{quotation}
}
\url{http://www.npr.org/blogs/health/2013/12/27/257552665/concussions-may-increase-}
\url{alzheimers-risk-but-only-for-some}

 Be prepared to answer questions about 
\begin{enumerate}
\item  The questions researchers wanted to answer.
\begin{students}
 \vspace{1in}
\end{students}

\begin{key}
  Are athletes exposed to brain trauma more at risk of Altzeimer's disease?
\end{key}


\item What were the subjects?
\begin{students}
 \vspace{1cm}
\end{students}

\begin{key}
 589 people involved in a long term brain study
\end{key}



\item Were treatments applied at random?
\begin{students}
 \vspace{1cm}
\end{students}

\begin{key}
No
\end{key}

\end{enumerate}
\end{enumerate}


