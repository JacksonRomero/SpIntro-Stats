\def\theTopic{Reading 5}

Title:  ``Empathy and Pro-Social Behavior in Rats''

Authors: Inbal Ben-Ami Bartal, Jean Decety, Peggy Mason

Journal: {\it Science} {\bf 9} December 2011: Vol. 334 no. 6061 pp. 1427-1430 


ABSTRACT\\
\begin{quotation}
  Whereas human pro-social behavior is often driven by empathic concern
for another, it is unclear whether nonprimate mammals experience a
similar motivational state. To test for empathically motivated
pro-social behavior in rodents, we placed a free rat in an arena with
a cagemate trapped in a restrainer. After several sessions, the free
rat learned to intentionally and quickly open the restrainer and free
the cagemate. Rats did not open empty or object-containing
restrainers. They freed cagemates even when social contact was
prevented. When liberating a cagemate was pitted against chocolate
contained within a second restrainer, rats opened both restrainers and
typically shared the chocolate. Thus, rats behave pro-socially in
response to a conspecific’s distress, providing strong evidence for
biological roots of empathically motivated helping behavior. 


\end{quotation}

\newpage

Watch this video:\\
\url{http://video.sciencemag.org/VideoLab/1310979895001/1/psychology}\vfill

Questions:
\begin{itemize}
\item What simple example of ``emotional contagion'' is mentioned? \vfill
 %  one baby cries -> all cry
\item What was the free rat's immediate reaction after first opening
  the cage door?  \vfill
\item What did both rats do when the caged rat was freed? (the first
  time). \vfill
\item How did the free rat's reaction change as it got used to the
  setup?  \vfill
\item Did the free rat open cages that contained:
  \begin{itemize}
  \item chocolates
  \item a toy rat
  \item nothing \vfill
  \end{itemize}
\item What does Peggy Mason conclude is ``in our brain''?\vspace*{\fill}
\end{itemize}