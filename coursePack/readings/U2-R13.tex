\def\theTopic{Reading 13}
\large
\begin{center}
  {\bf Hypothesis Test for a Single Mean}
\end{center}

Watch the video: Hypothesis Test for a Single Mean found:\\
\url{https://camtasia.msu.montana.edu/Relay/Files/j63r458/Single_Mean_Hypo_Test/Single_Mean_Hypo_Test_-_20151018_222435_23.html}

{\bf Questions}:
\begin{itemize}
  \item Why might people care about average snow depth in the
    mountains around Bozeman and whether or not it has changed
    recently?\vspace{2cm}
  \item Over the 30 years before 2011, what was average snow depth at
    Arch Falls?\vspace{2cm} %% 38 in
  \item Over the past 5 years, what has average snow depth at Arch
    Falls been? \vspace{2cm}%%34.7
  \item What variable is of interest? Is it quantitative or
    categorical? What statistic will we use to summarize it?\vspace{2cm}
  \item What are the null and alternative hypotheses?\vspace{2cm}
  \item Step 2 -- to create the null distribution -- uses a technique
    you've never seen before, and it's a bit weird. We'll work on it
    in the next class activity, so if you don't get it from the video,
    that's OK.\vspace{2cm}
  \item Aside from getting the simulated distribution, everything else
    should seem familiar. Why do we want the distribution to have the
    center it does?\vspace{2cm}
  \item How do we determine the p-value in this case? What is
    it?\vspace{2cm}  
  \item What do we conclude?\vspace*{\fill}
\end{itemize}

\normalsize
