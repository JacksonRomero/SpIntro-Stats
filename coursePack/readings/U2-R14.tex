\def\theTopic{Reading 14}


 Read this article about the justice system and the errors that are
 possible when trying a suspect.\\
\url{http://www.intuitor.com/statistics/T1T2Errors.html}  

\begin{center}
  {\bf Important points}
\end{center}

\begin{itemize}
  \item Assumption:
        \begin{itemize}
        \item In the justice system, what assumption is made about a
          defendant's guilt (or innocence)?\vfill
        \item The web page points out the similarities between the
          justice system and statistical hypothesis testing. What is
          the usual assumption in hypothesis testing (which parallels
          the justice system assumption above)?\vfill
        \end{itemize}
  \item Rejecting the assumption:
    \begin{itemize}
      \item In the justice system, what information causes a jury to
      reject the assumption they start with?  What is the standard for
      them to decide to reject it?\vfill
      \item In hypothesis testing, what information is used to reject
        the assumption?  How do we set the acceptable error rate? \vfill
      \end{itemize}
    \item Conclusions:
      \begin{itemize}
      \item In the justice system, if the jury decides it can reject
        the assumption, they find that the defendant is: \vspace{.8cm}
      \item In hypothesis testing that is equivalent to:\vspace{1cm}
      \item If the jury decides to reject the original assumption, they
        find the defendant: \vspace{.8cm}
      \item In hypothesis testing that is equivalent to:\vspace{1cm}
      \end{itemize}
  \end{itemize}
\newpage
    Their quality control example uses the assumption that a batch of
    some product is ``not defective''. Someone would test the batch to
    see if it has any problems, in which case the whole batch would be
    rejected. 

   We have not yet used the normal distribution which they show.  You
   can instead think of the distributions of dots from simulations you
   have seen in our web app. The idea of p-value is the same --
   that a statistic further out in the tail of the distribution gives
   a smaller p-value and is stronger evidence against $H_0$.

   \begin{itemize}
   \item Stonger evidence. Near the bottom of the web page they
     mention two ways we can reduce the chance of both type I and type
     II error. Here they are for the justice example.  Fill in the
     same ideas for the hypothesis testing situation.  
     \begin{itemize}
     \item More witnesses \vspace{1in}
     \item Higher quality testimony. \vspace{1in}
     \end{itemize}
   \end{itemize}
 
 You might like their applet which illustrates how we can reduce the
 chances of error, but it requires java which is not supported on Mac
 devices. 