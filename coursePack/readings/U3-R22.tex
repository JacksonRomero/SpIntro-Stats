\def\theTopic{Paired Intro }

\begin{center}
{\bf {\large Seeing Double}}\\ 
\end{center}

One of the assumptions for doing any statistical inference in this
course is that we have independent observations.  
We have also seen situations in which researchers obtained two
measurements on one unit:
\begin{enumerate}
  \item  When assessing the effect of energy drinks and alcohol on
    people, researchers measured each subject before and after they
    drank the beverage.  The two measurements on the same person are
    not independent because some people will tend to score high each
    time and others will tend to score low.
  \item Texting study ``B'' on page 191 of the last reading used each
    subject for the ST  condition and for DT condition.  
    \begin{enumerate}
    \item Are the two  observations independent?\vspace{1cm}
    \item If not, what characteristics of the person would affect both
      observations? \vspace{1cm}
    \end{enumerate}
  \item Similarly, study ``C'' on page 192-193 observed drivers on a
    driving simulator under several conditions (no texting, texting
    manually, texting with Google Glass).  The same issues with
    independence occur for this study as well.
  \end{enumerate}

It might sound like we don't recommend measuring people multiple
times, but actually, \vspace{.4cm}

{\bf it's a good idea to take multiple measurements on people.}\vspace{.4cm}
  

If you read the ``Energy Drinks'' study carefully, you'll see that we
did not analyze the RBANS scores individually, but subtracted post
score minus pre-score and analyzed ``Change in RBANS''.
\begin{itemize}
\item How many ``change in RBANS'' scores did we have for each
  person?\vspace{.5cm}
\item Are the ``change in RBANS'' measurement independent?\vspace{.5cm}
\end{itemize}

{\bf Take differences to get a single difference or change for each
  subject.} Then we have just one ``change'' per subject, and, as long
as one subject doesn't influence another, we can assume observations
are independent.  


Simple example:\\
  In the texting study ``B'' on page 191, each subject was tested
  under both DT (while texting) and ST (no texting) conditions.  We
  don't know why the authors specifically stated that they used an
  ``unpaired'' t-test, because this seems to be a situation where
  pairing should be used.  

  One should subtract one average reaction time from the other and
  analyze the differences as a one-sample t-test (or compute a
  one-sample t-based confidence interval) to make a statement about
  the mean difference, which can extend our inference to answer
  questions about the impact of texting while doing another task in a
  whole population of students.\vspace{.4cm}

The {\bf Paired T test} is really nothing new. 
\begin{itemize}
\item Take differences for each subject or unit.
\item Analyze the differences using one-sample t procedures. Degrees
  of freedom will be $n-1$, sample size minus one.
\end{itemize}
One caution: do be careful in stating results. 
Talk about the mean change or the mean difference, not the difference
in means (because we are using one sample methods, there is only one
mean, but it's the mean of the change scores or differences.)


{\bf Blocking}\\
  One of the principles of experimental design from page 73 is
  ``blocking'' whch means that we like to compare treatments on units
  which are as alike as possible.  The paired t-test uses each subject
  as one block, then condenses two observations per person down to a
  single difference.
 
{\bf Randomization}, another principle of experimental design, is
still important.  We wish that the texting study ``B'' had randomly
assigned  subjects to either ST or DT for their first experience
 instead of giving them all in the same order.  What lurking
variable  might change reaction times under the two conditions, but
could be ``evened out'' by randomly changing the order?\vspace{1cm}

A local company selling diet aides likes to advertise in the newspaper
showing pictures of someone who lost 75 pounds while using their
products.  
\begin{itemize}
 \item Should a person who would like to lose weight be heavily
   influenced by stories of one person's weight loss?\vspace{1cm}
 \item What evidence would be more impressive?\vspace{1cm}
 \item How would the ideas above be used to demonstrate the
   effectiveness of a dieting product?\vspace{1cm}
\end{itemize}



\begin{center}
  {\large\bf Important Points}
\end{center}
   
\begin{itemize}
  \item Know the assumptions for using $z$ and $t$ methods. \vfill
  \item It is a good idea to collect multiple observations on
    individuals under different treatment conditions, but before
    analyzing them, we should reduce them down to a single number per
    subject.\vfill
  \item We use one-sample t procedures to analyze differences. DF =
    $n-1$. \vspace*{1cm}
\end{itemize}

