\def\theTopic{Reading 6}



\begin{center}
\vspace*{-.2in}
{\bf {\large What Does ``Confidence'' Mean?}}\\
\end{center}

Mark Twain said:
\begin{quotation}
All you need in this life is ignorance and confidence, and then
success is sure.   
\end{quotation}

 from quarterback Joe Namath:
\begin{quotation}
When you have confidence, you can have a lot of fun. And when you have fun, you can do amazing things.  
\end{quotation}

and from scientist Marie Curie:
\begin{quotation}
  Life is not easy for any of us. But what of that? We must have
  perseverance and above all confidence in ourselves. We must believe
  that we are gifted for something and that this thing must be
  attained. 
\end{quotation}

The above quotes (from brainyquote.com) refer to  ``self confidence''
which is certainly important in any endeavor.
In statistics, the word ``confidence'' is best summarized as {\bf
  faith in the process} by which an estimate (in our case, an interval
estimate) was created.  A confidence interval carries information
about the {\bf location} of the parameter of interest, and tells us a lot
about the {\bf precision} of the estimate through the interval
length. 


In the news, interval estimates are often reported as a point value
and a {\bf margin of error}. 

\begin{quotation}
  71\% of Democrats and independents who lean to the Democratic Party
  say the Earth is warming due to human activity, compared with 27\%
  among their Republican counterparts (a difference of 44 percentage
  points). This report shows that these differences hold even when
  taking into account the differing characteristics of Democrats and
  Republicans, such as their different age and racial profiles. 
\end{quotation}

  Read the explanation from the  Pew Research Center  of how they
  conducted the poll,
\url{http://www.pewinternet.org/2015/07/01/appendix-a-about-the-general-public-survey-2/}.
 The margin of error they give is for what 95\% confidence level?
  \vfill

 How large is the margin of error for Republican/lean Republican?
\begin{students}
\vspace{.8cm}
\end{students}

\begin{key}
  {\em 5.1\%}
\end{key}

 For  Democrat/lean Democrat?
\begin{students}
\vspace{.8cm}
\end{students}

\begin{key}
  {\em 4.5\%}
\end{key}

\newpage

\begin{center}
  {\Large \bf Plus or Minus Confidence Intervals}
\end{center}

In the web app used in previous activities, we clicked on a confidence
level and the web app colored in the right number of dots as red to
put our selected percentage of sampled proportions in the center
(these stayed blue) and split the remainder into the two tails,
turning these more extreme points red.  We call this a ``percentile''
method because, for example, a 90\% CI has lower endpoint of the 5th
percentile and upper endpoint of the 95th percentile.

Another common way of building a 95\% confidence interval is to take
the estimated value and add and subtract twice the standard error of
the statistic.  A 95\% confidence interval for $p$ is then
 $$ \phat \pm 2 SE(\phat)$$
where $SE(\phat)$ is a number coming from the plot on the web app.
Why 2?  Well, it's easy to remember, and with a symmetric
distribution, 95\% of the data will fall within 2 SD's (standard
deviations) of the mean.

Margin of error is then the amount we add and subtract.  In this case,
it is twice $SE(\phat)$.  (Note: the parentheses do not mean
multiplication, say of SE times $\phat$. They indicate that $SE$ is a
function of $\phat$, in the same way we use $\log(x)$ or
$\sin(\theta)$.)\\
Open the web app: \url{https://jimrc.shinyapps.io/Sp-IntRoStats}.

\begin{enumerate}
\item Go back to the rat data from Activity 6 where 23 rats opened the
  cage and 7 did not.  Reenter the data in the \fbox{One Categ} part
  of the web app, and select \fbox{Estimate}. 
  \begin{enumerate}
  \item Generate 5000 to 10,000 resamples and click 95\%. Record the
    interval here:
\begin{students}
\vspace{1.5cm}
\end{students}

\begin{key}
  {\em (0.60, 0.90)}
\end{key}
\item Now write down the SE shown near the top right corner of the
  plot.  (We will not use the mean of the plotted values).
\begin{students}
\vspace{1.5cm}
\end{students}

\begin{key}
  {\em 0.077}
\end{key}
\item Add and subtract $2SE$ from the original proportion given in the
  box at left ( {\bf Do not} use the mean from the plot.) and write it
  in interval notation.

\begin{students}
\vspace{1.5cm}
\end{students}

\begin{key}
  {$ 0.77 \pm 2\times 0.077 =  (0.63, 0.91)$}
\end{key}
\item Compare the two intervals.  Is one wider? Is there a shift?

\begin{students}
\vspace{1.5cm}
\end{students}

\begin{key}
  {\em The percentile CI is shifted slightly to the left and is
    slightly wider.}
\end{key}

  \end{enumerate}
\end{enumerate}
