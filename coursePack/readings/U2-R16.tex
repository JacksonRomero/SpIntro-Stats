\def\theTopic{Reading 16}

Watch this video:

\url{https://camtasia.msu.montana.edu/Relay/Files/j63r458/Hypothesis_Test_for_Slope_and_Correlation/Hypothesis_Test_for_Slope_and_Correlation_-_20151025_221023_23.html}
\large

\begin{itemize}
 \item What is the usual null hypothesis for a test of slope? Be sure
   to use the right parameter. \vspace{2cm}

 \item What is the alternative hypothesis for the example of taxi tips?\vspace{2cm}

 \item What is the equation of the least squares line?\vspace{2cm}

 \item If you are heading to NYC, what percentage would you expect to give a
   cabbie as a tip?\\
    (These data are for fares paid with credit cards. If we look at
    fares paid with cash, lots of people gave no tip. Apparently
    people doing 'business lunches' tend to use cards more, and they
    can include the tip as part of the expense, so they are more
    generous.  Answer the question as if you are one of them.) \vspace{2cm}

  % \item Why is the p-value so small?  Compare what you see in the
  %   lower left hand plot with the upper left hand plot. Try clicking
  %   on different points in the right hand plot to see what data
  %   generated the slope seen on the right. \vspace{2cm}

  \item What is the usual null hypothesis for a test of correlation? Be sure
   to use the right parameter. \vspace{2cm}
  \item Explain the null hypothesis in your own words.  What does it
    say about the relationship between the two variables?\vspace{2cm}

  \item What is the alternative hypothesis for the example of taxi tips?\vspace{2cm}

\end{itemize}
\normalsize
