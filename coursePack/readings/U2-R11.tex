\def\theTopic{Reading 11}


\begin{center}
  {\large\bf Statistical Significance}
\end{center}

Disclaimer:  Most statisticians prefer to just report p-value as the
``strength of evidence'' and let readers evaluate the evidence against
$H_0$ for themselves.  Furthermore, if evidence from a study could be
summarized with a confidence interval, then that is a good way to
report results, too.  However, you will see results from many studies
summarized as ``statistically significant''  or as ``not statistically
significant'', so we need to talk about the meaning of those phrases.

{\bf Statistical significance} means
\begin{enumerate}
\item that researchers decided to use a  cutoff level for their study,
  typically 0.10 or 0.05 or 0.01, and
\item the p-value for the study was smaller (stronger evidence) than
  the chosen cutoff level.
\item researchers rejected the null hypothesis at the given $\alpha$
  level. 
\end{enumerate}

{\bf Significance level}, $\alpha$ is the chosen cutoff. The three
values listed above are most commonly used, but there is no reason
(other than an ingrained habit) that we could not use other levels.\\
How does one choose a significance level?
\begin{itemize}
  \item If very strong evidence is required (lives are at stake, or
    the null hypothesis has very strong support) then a small $\alpha$
    like 0.01 would be used.
  \item If we are just exploring a new area or the null hypothesis is
    not something people believe in, then weak evidence would be all
    that is needed to reject it, and we could use $\alpha = 0.10$,
\end{itemize}
 Note that people will have different opinions on the above.  That's
 why we would rather let a reader decide how strong the evidence needs
 to be.  In reports of results, we still want you to report the
 p-value. 

 {\bf Problems with fixed $\alpha$ significance testing}
 \begin{itemize}
 \item Suppose we decide to use $\alpha = 0.01$ and the p-value is
   0.0103.  We then ``Fail to reject'' at our set $\alpha$ level.  But
   someone else might repeat the study, get very similar results, and
   a p-value of 0.0098 which is less than $\alpha$, so they reject
   $H_0$.  If we just reported the p-value and let readers look at the
   strength of evidence, we would be able to say that the studies
   pretty much agree. With fixed level testing, we make quite
   different conclusions from very similar results.  That is a disturbing {\bf
     inconsistency}.
 \item P-values are strongly related to {\bf sample size}. Whether we
   ``reject'' or ``fail to reject'' depends as much on the sample size
   as it does   on the true state of nature. \\
   For example, the LEAP study of peanut allergies in the last
   activity used sample sizes of 245 and 255 infants and we had a
   p-value $< 0.0001$.  What if they had used one tenth the sample
   size as here:\\
\begin{tabular}[c]{|l|c|c|c|} \hline
        &Peanuts & Avoiders & total \\ \hline
Allergic&      0  &   4 &  4\\ \hline
Not Allergic& 24  &  22 &46 \\ \hline
Total &       24  &  26 & 50\\ \hline
\end{tabular}\\
   The proportions are almost the same, the difference in proportions
   is  $-0.15$ but the p-value is 0.066.\\
   Lessons: The researchers were smart to use a large sample
   size. However, if there was only a little difference in the two
   groups and they used a huge sample size, they would reject $p_1 =
   p_2$ when the difference in proportions allergic is very
   small. Which leads to our third point:
 \item Obtaining ``Statistical significance'' does not tell us
   anything about ``{\bf practical importance}''.  In common usage,
   the word significance  means something like ``importance''. For
   example, we talk about ``significant events'' in history. Consider
   these examples: 
   \begin{itemize}
   \item YouTube carefully examines how people navigate their web
     site.  Suppose they test two  web page designs (assigned at random) on 
     large groups of randomly selected viewers and find that there is
     a ``significant'' difference in mean time spent on their site
     (between the two designs) of 0.56 seconds ($\alpha$ was set to
     0.05). Is that an {\bf important}  difference?
\begin{key}
       \\ {\it It might be if it convinces advertisers to invest more
         in YouTube ads}.
\end{key}
   \item  Many businesses use a call center to answer questions from
     their customers.  They have to decide how many staff to have on
     hand to answer questions during business hours.  If they have too
     few people, wait times get long, and customers hang up without
     getting to a support team member. Suppose they have
     to decide between having 6 or 8 people and they do a test to
     measure the proportion of individuals who hang up before getting
     help. With 6 people the hangup proportion is 0.34 and with 8
     people it is 0.33.  Because they gathered data on several
     thousand customers, the p-value for the test is very small,
     0.002 which is ``statistically significant'' at the $\alpha =
     0.05$ level. Is a difference of 1\% of practical importance?
\begin{key}
       \\ {\it I would think not -- that they should look for other
         ways to improve service -- but it might seem important to the company.}
\end{key}
 
   \end{itemize}\vfill
 \end{itemize}



{\bf Important Points}
\begin{itemize}
\item Remember: smaller p-values provide stronger evidence against the
  null.  When we set a {\bf significance level}, we reject $H_0$ for
  p-values {\bf smaller} than $\alpha$.
\item Still report p-values.
\item Statistical significance $\neq$ practical importance.
\item When we do use a fixed $\alpha$ significance level, we will say
  either that we ``reject $H_0$'' (because p-value  $ \leq \alpha$) or
  that we ``Fail to reject $H_0$'' (when p-value $> \alpha$). We never
  accept the null hypothesis, but we can reject it.  
\end{itemize}
