\def\theTopic{Reading 21 }

\begin{center}
{\large\bf  Which tool to use?}
\\
\end{center}

In Unit 3 so far we have used the following tools:
\begin{enumerate}
\item Normal-based tests on a single proportion.
\item Normal-based confidence interval estimates of a single proportion.
\item Normal-based tests for equivalence of two proportions.
\item Normal-based confidence intervals to estimate the difference
  between two proportions.
\item t-based tests on a single mean
\item t-based CI's to estimate a single mean
\item t-based tests for the equality of two means  (coming up next)
\item t-based CI's to estimate a difference in means (coming up next)
\end{enumerate}

Below are reports of three studies. You will be asked to find the
appropriate tool for each situation.

\begin{alist}
\item ``Effects of Insulin in Relatives of Patients with Type 1
  Diabetes Mellitus'' by the Diabetes Prevention Trial–Type 1 Diabetes
  Study Group published in {\it N Engl J Med} 2002; 346:1685-1691

  \begin{list}{}{}
  \item [\bf Background]
    It is unknown whether insulin therapy can delay or prevent
    diabetes in nondiabetic relatives of patients with diabetes. 
  \item [\bf Methods]
    In a randomized, controlled, nonblinded clinical trial, we
    screened 84,228 first-degree and second-degree relatives of
    patients with diabetes for islet-cell antibodies and identified
    339 who had a projected five-year risk of becoming diabetic of
    more than 50 percent; they were randomly assigned to undergo
    either close observation (n = 170) or an intervention (n = 169)
    that consisted of low-dose subcutaneous ultralente insulin,
    administered twice daily for a total dose of 0.25 unit per
    kilogram of body weight per day, plus annual four-day continuous
    intravenous infusions of insulin. Oral glucose-tolerance tests
    were performed every six months. Median follow-up was 3.7 years.
  \item [\bf Results]
    Diabetes was diagnosed in 69 subjects in the intervention group
    and 70 subjects in the observation group. 
  \end{list}
  \begin{enumerate}
   \item What is the researchers' question?\vspace{1cm}
   \item What two variables will be used in the analysis?\vspace{1cm}
   \item What is the parameter of interest?\vspace{1cm}
   \item Which two of the tools listed above are appropriate for these
     data? Explain why.\vspace{2cm}
   \item Just based on the reported results, do you expect a test to
     give a large or small p-value?  Why?\vspace{1cm}
   \item Again, without doing the analysis, we would expect a 90\%
     confidence interval for some parameter to contain zero.  What
     parameter? \vspace{1cm}
  \end{enumerate}
% \item [Debunking the Myths Commonly Believed to Affect Test Performance
%   among College Students]
\item ``The Effects of Text Messaging During Dual-Task Driving Simulation on Cardiovascular and Respiratory Responses and Reaction Time''
  by:  	Park, Andrew; Salsbury, Joshua; Corbett, Keira; Aiello,
  Jennifer. 
  from: {\it Ohio J Sci} 111 (2-5): 42-44 

  \begin{list}{}{}
  \item [\bf Abstract]
    Research over the past decade has shown the potentially harmful
    effects of distracted driving, particularly on reaction time of
    the driver to external stimuli. With the recent surge in frequency
    of the use of cell phones for text messaging in nearly all
    situations, including during driving, it is important to
    understand the impact of texting on driver reaction time and the
    body’s physiological response. This study attempts to replicate
    the effects of text messaging distractions on reaction times found
    in previous studies.
  \item [\bf Methods]
    The 40 subjects in this IRB approved study were students at Ohio
    Northern University between the ages of 18 and 22 who used their
    cell phones daily for texting purposes. All subjects refrained
    from caffeine and alcohol consumption for a minimum of 24 hours
    prior to testing.

    ST (single task): Each was seated with both hands next to but not
    touching the 
    computer mouse and forearms resting on the desk. The subject would
    click the mouse each time the stoplight changed from red to
    green. Subjects would then reposition their arms to the original
    position. The test allowed for five reaction time (RT) trials to be taken
    consecutively over the course of approximately one minute with
    random rest intervals as determined by the test.  The test then
    automatically calculated mean RT in seconds (s).

    DT (dual task): After a five-minute rest period, subjects repeated
    the RT test with dual-task conditions. Each participant was seated
    with forearms resting in the same manner, but with both hands
    holding their phone next to but not touching the mouse. A typed
    sheet of paper with questions asking for the participant’s full
    name, age, address, and contents of last meal was taped to the top
    of the computer screen, but not obstructing the participant’s view
    of the stoplight.  This position simulated the shift in visual
    attention a driver would experience while attempting to text and
    drive. Participants were not allowed to read the questions prior
    to testing. Once the test began, subjects were to read the
    questions and respond to as many as possible in a texted message
    while maintaining visual focus on his or her phone while
    texting. This was done while simultaneously completing the
    reaction test, requiring participants to remove one hand from the
    phone to click the mouse in response to stimuli. Again, five
    trials were taken and the mean recorded as his/her dual-task (DT)
    reaction time.  Means for all participants in ST and DT conditions
    were compared using an unpaired two-tailed test (N=40).
  \item [\bf  Results]\ \ \\
   ST mean = 0.51 sec, SD = 0.41 sec\\
   DT mean = 1.22 sec, SD = 0.36 sec
 \end{list}
   \begin{enumerate}
   \item What is the researchers' question?\vspace{1cm}
   \item What two variables will be used in the analysis?\vspace{1cm}
   \item What is the parameter of interest?\vspace{1cm}
   \item Which two of the tools listed above are appropriate for these
     data? Explain why.\vspace{2cm}
   \item Just based on the reported results, do you expect a test to
     give a large or small p-value?  Why?\vspace{1cm}
   \item Again, without doing the analysis, we would guess that a 90\%
     confidence interval for some parameter would not contain zero.  What
     parameter? \vspace{1cm}
  \end{enumerate}

\item ``Texting while driving using Google Glass$^{TM}$: Promising but not
  distraction-free.'' by: He, Jibo, et al. in: {\it Accident Analysis
    \& Prevention} {\bf 81} (2015): 218-229.\\
  \begin{list}{}{}
  \item [\bf Abstract] Texting while driving is risky but common. This
    study evaluated how texting using a Head-Mounted Display, Google
    Glass, impacts driving performance. Experienced drivers performed
    a classic car-following task while using three different
    interfaces to text: fully manual interaction with a head-down
    smartphone, vocal interaction with a smartphone, and vocal
    interaction with Google Glass. Fully manual interaction produced
    worse driving performance than either of the other interaction
    methods, leading to more lane excursions and variable vehicle
    control, and higher workload. Compared to texting vocally with a
    smartphone, texting using Google Glass produced fewer lane
    excursions, more braking responses, and lower workload. All forms
    of texting impaired driving performance compared to undistracted
    driving. These results imply that the use of Google Glass for
    texting impairs driving, but its Head-Mounted Display
    configuration and speech recognition technology may be safer than
    texting using a smartphone.
  \item [\bf Highlights]
    \begin{itemize}
    \item Texting using Head-Mounted and Head-Down Displays both
      impair driving performance.
    \item Texting vocally is less disruptive to driving than texting manually.
    \item Texting using Head-Mounted Display is less distracting than Head-Down Display.
    \item Speech recognition and Head-Mounted Display can potentially reduce distractions.
    \end{itemize}
  \item [\bf Methods] Twenty-five students (12 males and 13 females; M
    = 20.48 years, SD = 2.14 years, ages range from 18 to 25 years)
    from Wichita State University participated in the driving
    experiment for course credit.
  
    This study used a classic car following task on a driving
    simulator. One recorded response was how well the driver stayed in
    the lane as measured by the number of incursions into another lane.
  \end{list}
  Suppose that you are a researcher in driving safety and already have
  lots of data about how well people perform this task when not
  texting. In fact, we will assume the mean for the non-texting
  population is 3.1.  You want to know if this mean also holds for
  people who are texting through Google Glass.
   \begin{enumerate}
   \item What one variable will be used in the analysis?\vspace{1cm}
   \item What is the unknown parameter of interest?\vspace{1cm}
   \item Which two of the tools listed above are appropriate for these
     data? Explain why.\vspace{2cm}
   \item Of the two possible tools, Which will more directly answer
     the research question? Why? \vspace{1cm}
  \end{enumerate}

\end{alist}






