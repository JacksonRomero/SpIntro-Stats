\def\theTopic{Energy Drinks II}
\def\dayNum{25 }

\begin{center}
{\bf {\large \textbf{More Energy Drinks}}}
\end{center}

Back on Activity 14 we compared an energy drink with
alcohol, REDA, to a control.  Our conclusion that ``change in RBANS''
was lower in the REDA group did not allow us to say whether that was
due to the alcohol 
or the stimulant in REDA.  The two  explanatory variables were
``confounded'' meaning that we can't separate the effects.  The
researchers knew this would be a problem, so they included a third
group in the study: 9 randomly selected women got RED, an energy drink
with no alcohol.  Today we'll compare RED and control means using
confidence intervals and hypothesis tests based on a $t$
distribution. 

  The research question is: 

  {\sf
    Does neuropsychological performance (as measured on the RBANS
    test) change after drinking an energy drink? }

   Higher RBANs scores indicate better memory skills.

 Go to the usual website and use the pre-loaded {\tt REDvsControl}
 dataset under \fbox{One of Each}.

  \begin{enumerate}
  \item  Obtain ``Actual'' Mean and Std dev for each group. 
\begin{students}
    \vspace{4cm}    
\end{students}

\begin{key}
  {\it Means:  control: 1.219, RED: -2.44 \\%, REDA:  -8.39\\
       Std.dev.: control: 3.919, RED: 6.425}%, REDA, 10.02} 
\end{key}

\item Using t-based methods requires that we either have ``near
  normal'' data or  large sample sizes.  Do you see extreme outliers
  in the plots?  Are the data skewed?

\begin{students}
    \vspace{2cm}    
\end{students}

\begin{key}
  {\it  These look OK to me, though the spreads do not look equal.}
\end{key}

\end{enumerate}


We'll start by comparing RED to Control  using a hypothesis test.

%\newpage
\begin{enumerate}
\setcounter{enumi}{2}

\item  What are the null and alternative hypotheses when comparing RED
  to Control? Use notation, and write them out with words.
  Do not assume they knew ahead of time which  mean would be
  larger. \\
\begin{students}
 $H_0:$    \vspace{1cm}    \\ $H_a:$    \vspace{1cm}
\end{students}
\begin{key}
  {\it   $H_0:\ \mu_1 = \mu_2$ The mean change in RBANS is the same
    for treatment (RED) and Control.    \\ $H_a: \mu_1 \neq \mu_2$ The
    mean change in RBANS is not the same for treatment (RED) and control.  }
\end{key}

   Check with another group to be sure we have the correct
   direction for $H_a$.

\item  Let group 1 be Control and group 2 be RED. Compute $SE(\xb_1 -
  \xb_2) = \sqrt{\frac{s_1^2}{n_1} +     \frac{s_2^2}{n_2}}$. 
\begin{students}
    \vspace{2cm}    
\end{students}

\begin{key}
  {\it  $ \sqrt{\frac{3.92^2}{9} +    \frac{6.43^2}{9}} = \sqrt{1.707
    +4.594} = 2.5102$.} 
\end{key}
\item As with a single mean, our test statistic is called $t$ and is
  found by dividing  the difference between sample statistic and its
  value under $H_0$ by its $SE.$  Find t. \\
\begin{students}
  $ t = \frac{\xb_1 - \xb_2 - (\mu_1 - \mu_2)}{SE(\xb_1-\xb_2)} = $
    \vspace{2cm}    
\end{students}

\begin{key}
   $ t = \frac{\xb_1 - \xb_2 - \mu_1 - \mu_2}{SE(\xb_1-\xb_2)} = 
          \frac{ 1.219-( -2.44)}{2.5102} = 1.458 $
\end{key}

\item We need to figure out which degrees of freedom for our t
  test. When working with two means, we take the smaller sample size
  and subtract one from it. What is that for the Energy Drinks study?
  ({\small This is a conservative approach. Other books or software
    programs might use a different number.})
\begin{students}
    \vspace{2.6cm}    
\end{students}

\begin{key}
  {\it  8}
\end{key}

 \item  Go to the t distribution part of the applet and put your t
   test statistic in the top box, 
   set its degrees of freedom, and select the right direction for the
   alternative.  What is our p-value?  State your decision and 
   conclusion about these two drinks. 
\begin{students}
    \vspace{5cm}    
\end{students}

\begin{key}
  {\it 0.182. At any of the commonly used $\alpha$ levels, we fail to
    reject the null hypothesis that mean change in RBANS score is the
    same for control and RED groups.  There is not sufficient evidence
    to say that they differ.}
\end{key}


 \item  Next we want to compare RED drinkers to REDA drinkers (group 2
   to group 3). 

   \begin{enumerate}
     \item Your null and alternative hypotheses are the same, but we
       need to use subscripts 2 and 3 instead of 1 and 2.  Copy them
       here. 
\begin{students}
    \vspace{2cm}    
\end{students}

\begin{key}
  {\it   $H_0:\ \mu_2 = \mu_3$ The mean change in RBANS is the same
    for RED and REDA.    \\ $H_a: \mu_2 \neq \mu_3$ The
    mean change in RBANS is not the same for RED and REDA.  }
\end{key}
 

      \item Compute the difference in means.
\begin{students}
    \vspace{1cm}    
\end{students}

\begin{key}
  {\it  $ -2.44 -( -8.329) = 5.889$}
\end{key}

      \item Compute the $SE$ of the estimator.
\begin{students}
    \vspace{1cm}    
\end{students}

\begin{key}
  {\it  $ \sqrt{ \frac{6.43^2}{9} + \frac{10.02^2}{9}} = \sqrt{
    4.594 + 11.1566 } = 3.969$ }
\end{key}

      \item Compute the $t$ test statistic.
\begin{students}
    \vspace{1cm}    
\end{students}

\begin{key}
  {\it   $ 5.889 / 3.969 = 1.484$}
\end{key}

      \item What degrees of freedom will you use?  Find the p-value.
\begin{students}
    \vspace{1cm}    
\end{students}

\begin{key}
  {\it  8 df.  The p-value is $2 \times 0.088 = 0.172 $}
\end{key}

      \item What is your decision? your conclusion? 
\begin{students}
    \vspace{2cm}    
\end{students}

\begin{key}
  {\it  Again, we do not have much evidence against the null
    hypothesis that the two groups have the same mean. We fail to
    reject $H_0$ at all commonly used $\alpha$ levels. }
\end{key}

     \end{enumerate}

     \begin{center}
       {\large\bf Confidence Interval}

 Reminder. The general form of a t-based CI is:
     \end{center}
  $$ \mbox{estimate} \pm t^*_{df} \times SE(\mbox{estimate}) $$ 

 \item Finally, we'll go back to the original comparison between REDA
   and Control means. However, we'll set this one up as a confidence
   interval instead of a t-test. 
   \begin{enumerate}
   \item Compute the difference in means between REDA and control.
\begin{students}
    \vspace{1cm}    
\end{students}

\begin{key}
  {\it  $1.219-( -8.329) = 9.548 $}
\end{key}

   \item Compute  $SE(\xb_1-\xb_3)$
\begin{students}
    \vspace{1cm}    
\end{students}

\begin{key}
  {\it $\sqrt{ \frac{3.919^2}{9} + \frac{10.02^2}{9}}= 3.587$}
\end{key}

   \item What degrees of freedom do we need?\\
     Use the web app to find the $t^*$ multiplier for a 95\% 
     confidence interval. 
\begin{students}
    \vspace{1cm}    
\end{students}

\begin{key}
  {\it  8 df.   $t^* = 2.306 $}
\end{key}
\begin{students}
    \vspace{1cm}    
\end{students}

\begin{key}
  {\it  8 df.   $t^* = 2.306 $}
\end{key}

   \item Find the margin of error and construct the 95\% CI.  Does it
     contain zero?  If testing 
     $H_0: \mu_1 = \mu_3$ versus a two-sided alternative at the
     $\alpha = 0.05$ level, would you reject $H_0$?  Explain.
\begin{students}
    \vspace{1cm}    
\end{students}

\begin{key}
  {\it  ME $=2.306 \times 3.587 =8.272 $, 95\% CI:   $9.548 \pm  
    8.272 = ( 1.28, 17.82)$ Zero is not in the interval, so it is not
    a ``plausible value'' for the difference in means.  We would
    reject the null in favor of the alternative at $\alpha = 0.05$.  }
\end{key}

   \end{enumerate}

 \item   How do you explain the fact that two of the three
   t-tests we've done gave large p-values and another gave a small p-value? Is
   that inconsistent? 
\begin{students}
    \vspace{3cm}    
\end{students}

\begin{key}
  {\it  The confidence interval shows that we have
    fairly strong evidence that the mean for control change in RBANs is
    larger than the mean for REDA change in RBANS. We do not have very
    large sample sizes in this study, so it is not surprising that two
    comparisons found that the middle mean (RED) was not very
    different from either extreme (Control or REDA), yet the two
    extreme values were far enough apart to detect a fairly strong
    difference. }
\end{key}

\item Could the difference in REDA and Control means be just due to
  random chance?  Explain.
\begin{students}
    \vspace{2cm}    
\end{students}

\begin{key}
  {\it Yes.  Our conclusions do not change just because we've used a
    different method to compute the p-value.  It's still possible that
  one group was higher just due to random assignment of treatment.}
\end{key}
\begin{students}
    \vspace{2cm}    
\end{students}

\begin{key}
  {\it Yes.  Our conclusions do not change just because we've used a
    different method to compute the p-value.  It's still possible that
  one group was higher just due to random assignment of treatment.}
\end{key}
\item Can we make causal inference about the effects of energy drinks
  and alcohol?
\begin{students}
    \vspace{2cm}    
\end{students}

\begin{key}
  {\it  Yes.  Because treatments were randomly assigned, it is very
    unlikely that we would see such large differences just by chance,
    so the observed differences (or lack thereof) are attributable to
    the treatments.  }
\end{key}


\item Write up the hypothesis test results as a report.  Include all
  three comparisons we've made. \vfill
   
 \end{enumerate}
 \newpage

\begin{center}
  {\large\bf Take Home Message}
\end{center}

\begin{itemize}
  \item Interpretation of results does not change just because we
    switched from permutation testing to t-tests.  We still ask ``Was
    this a random sample of subjects?'' to obtain inference back to a
    population, and we still ask ``Were treatments assigned at
    random?'' to conclude that change in one variable caused changes
    in the other. 
  \item The t-test approach uses formulas for standard errors, while
    the permutation test relies on repeated permutations or resampling
    to get the same information about the spread of the sampling
    distribution under $H_0$.  Both work!
  \item We can use t-tests to compare two means, but do have to be
    careful with small sample sizes.  We should not use t-tests with
    sample sizes less than 15 unless the data look symmetric with no
    outliers.
  \item With large sample sizes, t methods are robust to outliers and
    skewness.
  \item When sample sizes are small, only {\em large} differences will
    be flagged as having ``strong evidence''.  We'll look at this
    later in more detail.
  \item What questions do you have?  Write them here.
\end{itemize}


%% 4 pages