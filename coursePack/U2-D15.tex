\def\theTopic{Birth Weights }
\def\dayNum{15 }

\begin{center}
\vspace*{.1in}
{\bf {\large Birth Weights}}
\end{center}
\vspace{-.1in}

Lab tests with animals have shown that exposure to tobacco smoke is
harmful in many ways.  To make connections to humans has been more of
a challenge.  One dataset which might help us connect tobacco use of
pregnant women to birth weights of their babies comes from a large set
of data on {\bf all} births in North Carolina.  We will examine a random sample
of size 200 from the much larger dataset. The two variables provided
are {\tt habit} (either smoker or nonsmoker) and {\tt weight} (baby's weight at
birth measured in pounds). 



{\bf Discuss}
\begin{enumerate}
\item   Could there be some physiological reason why birthweights for
  the children of the 28 smokers might differ from the birth weights
  of the babies born to nonsmokers?  Write down what you and your
  group know about smoke and nicotine to hypothesize a connection to
  birthweight. 
\begin{students}
  \vspace{2cm}
\end{students}


\item   If the connection you are thinking about is real, would it
  tend to increase or decrease birth weights of babies born to
  smokers?  Or could the effect go either way?
\begin{students}
  \vspace{1cm}
\end{students}

\begin{enumerate}
\item   What is the response variable in this study?  Is it
  quantitative or categorical?
\begin{students}
  \vspace{1cm}
\end{students}

\begin{key}
  {\it birth weight, quantitative}
\end{key}
  
\item  Is there an explanatory variable in this study?  If so, name it
  and tell which type of variable it is.
\begin{students}
\vspace{1.5cm}
\end{students}

\begin{key}
  {\it Yes, habit, and it  is categorical}
\end{key}

  
\item Select the ``Pre-loaded'' data (birthweights)  from the
  \fbox{One of Each} menu at 
  \url{http://shiny.math.montana.edu/jimrc/IntroStatShinyApps}
   Compute means for weight by habit and compute the difference in means.
\begin{students}
\vspace{2cm}
\end{students}

\begin{key}
  {\it Smoker: 6.306, nonsmoker: 7.084, difference: -.778 }
\end{key}

   
\item  Is the difference between the means large enough to convince
  you that babies born to smoking mothers are lighter than those born
  to nonsmokers?
  Why or why not?
\begin{students}
 \vspace{3cm}
\end{students}

\begin{key}
  {\it It does not seem like a difference of .78 is that large because of
the spread of the plots (range from 1.7 to 8 for for smokers, 1.4 to
10 for nonsmokers). }
\end{key}

\end{enumerate}
\end{enumerate}

{\bf Studies that Use Random Sampling}

 The big differences between this study and the previous studies where
 you compared two conditions is the subjects in this study were a
 {\bf random sample} from a larger population. The use of random sampling
 versus the use of random assignment  changes the type of
 inferences that can be made. 

A random sample is one in which the method used to choose the sample
from the population of interest is based on chance.

Reminder: When studies employ random {\bf assignment}, we are able to draw
cause--and--effect conclusions about the {\bf treatment effects}. 
Randomization evens out the influences of all possible lurking
variables, and allows us to conclude that treatments really made a difference.  


With data on birth weights, can we assign a baby to have  a smoking
versus nonsmoking mother?  \vspace{1cm}


The habit variable 
splits these subjects into two populations, and we have a sample
from each.  In studies with random {\bf sampling}, the goal is to
describe the sample data, to compare groups, and infer any differences
back to the broader population(s) from which the sample(s) was/were
drawn.  Even though we might have reasons from animal studies to think
smoking causes certain changes, we do not expect these data to provide
{\bf causal} evidence of such a connection.  We might want to know how large the
difference in means (the true population means) is between two groups.
That's really an estimation question. 

In the  peanut allergy study we found strong evidence of a
difference due to the therapy, but the inference only applied to the
kids in the study because we could not say that the sample of
subjects was representative of a larger population, like all infants.

An ideal study would start with a random sample from the population of
interest so that we can make inference back to the population, and it
would use random treatment allocation to allow causal inference. In
practice, we must often settle for a convenience sample, so our
inference only extends back to a subset of the population.   


{\bf MODELING THE BIRTH WEIGHTS}

You will conduct a {\bf permutation} test to find out how likely it would
be to see this large a difference in sample means if the two
populations really have the same overall mean birth weight. The word
``permutation''  emphasizes that we could be assigning treatments, or just
shuffling labels we observed from different groups.  By doing the
relabeling many times, we can see what results are expected when the
populations really have the same distribution of responses. 

\begin{enumerate}
  \setcounter{enumi}{5}
  \item  Describe the null model to be used to simulate data in this
    investigation. 
\begin{students}
\vspace{2cm}
\end{students}

\begin{key}
  {\it  The mother's habit of smoking or not is not
    associated with weight of baby. Or baby's weights are the same, on
    average, for smoking and nonsmoking mothers. }
\end{key}


    Copy the means of each group and the difference in means from \# 2
    here.
   
\begin{students}
\vspace{1cm}  
\end{students}

\begin{key}
  {\it  smoking: 6.31, nonsmoking: 7.08, difference: 0.778 }
\end{key}


\item  Check results for the first resample. What is the mean birth weight to
    smokers from this     simulated trial?  
\begin{students}
 \vspace{1cm}
\end{students}

\begin{key}
  {\it Answers will vary,  I got 6.5}
\end{key}

    
    What is the mean  birth weight to
    nonsmokers  for this single simulated trial?
\begin{students}
 \vspace{1cm}  
\end{students}

\begin{key}
  {\it 7.05}
\end{key}
            

    What is the difference in means between these two groups?
\begin{students}
 \vspace{1cm}
\end{students}

\begin{key}
  {\it 0.53}
\end{key}

           

{\bf Evaluate the Results }

\item  Plot the differences in means from  1000 or more simulated
  trials. Sketch the plot below.
\begin{students}
 \vspace{4cm}
\end{students}

\begin{key}
  {\it }
\end{key}



\item  What does each dot in the plot represent? % What changes when you
%  click on a bar? 
\begin{students}
 \vspace{1cm}
\end{students}

\begin{key}
  {\it Each dot represents 1 re--randomized trial where the responses
   were kept the same but the habit for the responses
   was randomized (this represents what could happen under the null
   hypothesis that habit is not associated with
   birth weight).  Clicking on a bar changes the
   plot of the “Most Recent Shuffle” showing you which
   re--randomized trial that bar represents.  The placement
   of the point gives the difference in the mean birth
   weight between the two habits. }
\end{key}

   
\item  Where is the plot of the results centered (at which value)?
  Explain why this makes sense.
\begin{students}
 \vspace{3cm}
\end{students}

\begin{key}
  {\it  Centered at 0 because this is showing us what could happen if the
  null model were true and the null hypothesis says there should be
  no difference in mean birth weight for smokers and nonsmokers. }
\end{key}


\item  We're not told exactly what the researchers were thinking ahead
  of time, but let's assume that the alternative hypothesis is that
  smoking moms tend to have lighter babies.   What is the alternative
  hypothesis of interest? Do you need to count \fbox{Greater} than or
  \fbox{Less}  than or \fbox{more extreme} results to find the p-value? 
 
\begin{students}
 \vspace{1cm}  
\end{students}

\begin{key}
  {\it  The mean birth weight for babies whose mother smoked is lower
  than the mean birth weight for babies whose mothers did not
  smoke. Use less than -0.778}
\end{key}

 
\item   Put the observed difference in the little box under the plot,
  chose the proper direction for comparison, and report the
  approximate p--value (i.e., strength of evidence) 
  based on the observed result. 
\begin{students}
 \vspace{1cm}
\end{students}

\begin{key}
  {\it  0.013}
\end{key}

  %% 
\item  Based on the p--value, how strong would you consider the
  evidence against the null model?  
\begin{students}
 \vspace{1cm}
\end{students}

\begin{key}
  {\it  Strong  to very strong evidence against the null model.}
\end{key}

   
\item  Based on the p--value, provide an answer to the research
  question.  
\begin{students}
 \vspace{1cm}
\end{students}

\begin{key}
  {\it There is strong evidence against the null hypothesis
  that smoking habit is not associated with mean birth weight.  We can
  conclude there is an association 
  between these variables and that the mean birth weight
   is higher for babies with nonsmoking moms than for babies with
   smoking moms.}
\end{key}

  

\item  Can the researchers generalize the results to the population of
  all births in North Carolina?  to all births in the US? Why or why not?
\begin{students}
 \vspace{1cm} 
\end{students}

\begin{key}
  {\it  Yes because this is a random sample of all NC births, no
    because we didn't look at other states.}
\end{key}


\item Can the researchers say that the difference in the average
  birth weight is caused by the mother's smoking habit?  Explain. If
  not, provide an alternative   explanation for the differences. 
\begin{students}
 \vspace{2cm}
\end{students}

\begin{key}
  {\it   No because there was no random assignment of participants to
    a habit.  (The researchers did not randomly assign smoking to some
    moms and nonsmoking to others. Possible alternative explanations
    are listed in \#5.} 
\end{key}


\item  Write--up the results of the simulation study. When reporting
  the results of a simulation study, pertinent details from the
  analysis that need to be included are: (as in Activity 12)
  \begin{itemize}
  \item The {\it type of test} used in the analysis (including the number of trials);
  \item The {\it null model} assumed in the test;
  \item The {\it observed result} based on the data;
  \item The {\it p--value} for the test, whether it is one or two sided; and
  \item The {\it scope of inference} based on the p--value and study design.
  \end{itemize}
\begin{students}
\newpage
\end{students}
\begin{key}
  {\it A permutation (or randomization) test for a difference in
    proportions with 1000 shuffles was used to test the null
    hypothesis that birth weight is not associated with mothers
    smoking (or not).  In a random sample of 200 births, the mean
    birth weight was 6.31 lbs for babies whose mother smoked and 7.08
    lbs for babies whose mothers did not smoke. This gave an observed
    difference in means of $0.778$ (nonsmoking -- smoking).  This
    resulted in a p--value of 0.013, which constitutes moderate to
    strong evidence against the null hypothesis.  Since births were
    randomly sampled from all North Carolina but were not randomly
    assigned to a habit, we can conclude there is an association
    between these variables and that the mean birth weight really is
    lower for all births to smoking mothers than to nonsmoking
    mothers. }
\end{key}
\ 
\vspace*{\fill}

\item Finally, use the web app to create a 99\% bootstrap percentile
  interval estimate of the  difference in true mean babies weights
  from non-smoking to smoking mothers.  
  \begin{enumerate}

\item Obtain a 99\% bootstrap percentile interval by generating 5000
  samples and write it here.
\begin{students}
    \vspace{2cm}    
\end{students}

\begin{key}
  {\it (0.063, 1.59) lbs  }
\end{key}
    \item Write your interpretation of this interval.
\begin{students}
    \vfill
\end{students}

\begin{key}
  {\it  We are 99\% confident that the true mean birth weight for
    babies born to nonsmoking Moms is   .06 to 1.59 lbs greater than 
   the true mean birth weight of babies born to smoking mothers. Be
   sure to indicate which one is higher  (or lower). } 
\end{key}


\item To write up a report on a confidence interval, we must include:
  \begin{itemize}
  \item The observed result based on the data;
  \item The confidence level and the interval;
  \item The method used to create the interval estimate (for
        bootstrap include the number of resamples);
  \item The interpretation in the context of the data collected.
\end{itemize}
\end{enumerate}
\end{enumerate}

\begin{students}
  \newpage
\end{students}

\begin{center}
  {\bf Take Home Messages:}
    \vspace{-.4cm}   
\end{center}
\begin{itemize}
  \item In an {\bf observational study}, no treatment is applied,
    different groups are just observed.
  \item If we have a random sample from a population, we can infer
    results back to the population from which the subjects were drawn.
  \item When we do not randomly  assign treatments, we cannot be sure
    that there are no lurking variables. Therefore, other explanations
    for the observed results are possible, and we cannot infer a {\bf
      causal} relationship between our explanatory and response
    variables. It is just an association. \\
    You may have heard the term ``correlation does not imply
    causation'' used in cases like this. It's not quite accurate
    because (wait for Unit 3) correlation is a term for linear
    association between two quantitative variables.  Here we have a
    categorical explanatory variable and quantitative response.
  \item 
    It is possible that a causal effect exists, for example, we know
    that nicotine is a vaso-constrictor and poorer blood flow could
    reduce babies weights.  This is a stats class, so you are learning
    how far statistics can go with inference.  It is always possible
    to go further based on other natural laws or explanations, or
    laboratory results on simpler systems.  It is important to use
    other information, and we want you to distinguish stat inference
    from other ways of learning about the world.
 % \item We used StatKey to create a Bootstrap interval without
 %   explaining just how it does that.  The next activity explains the
 %   bootstrap in more detail.
 \item 
  Use the remaining space for any questions or your own summary of the
  lesson. 

\end{itemize}




