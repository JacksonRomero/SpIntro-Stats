\def\theTopic{Unit 2 Wrapup }
\def\dayNum{20 }

\begin{center}
\vspace*{-.2in}
{\bf {\large Unit 2 Wrapup}}\\
Vocabulary
\end{center}
\vspace{-.1in}
\begin{itemize}
  \item  Response and Explanatory variables
  \item  Random Assignment (why do we do it?)
  \item  Random Sampling
  \item   Lurking Variables
  \item  Causal inference (versus just association)
  \item Scope of Inference
  \item  Permuting labels, permutation test, randomization test
  \item  Bootstrap process: CI for $\mu$\\
         Percentile method\\
         estimate $\pm t^* SE$
  \item  What points must be included in a statistical report?
  \item  Stat significance is not the same as importance or practical
    significance.
  \item Interpretation of Confidence Interval
  \item Correlation, Slope
  \item  Type I Error  – probability is limited to $\alpha$
  \item  Type II Error is called $\beta$.  Power = $1-\beta$.
  \item  What settings affect power of a study?
  \end{itemize}
  
  We have built confidence intervals and done hypothesis tests for one
  mean, difference in proportions, difference in two means. And we did
  hypothesis testing for a slope (or correlation) being 0. (Could also
  estimate slope with a CI, but didn't have time).

  \begin{enumerate}
  \item For all studies in Unit 2 consider whether the study was an
    experiment or observational study.  What was the explanatory
    variable? the response?\\
 \begin{tabular}{|l|c|c|c|}\hline
Study&Experiment?&Explanatory Vble&Response Vble\\ \hline
Study Music &
\begin{key}
  Exp
\end{key}
&
\begin{key}
  Music or Quiet
\end{key}
&
\begin{key}
  SAT score
\end{key}
\\ \hline
Book Cost &
\begin{key}
  Obs
\end{key}
&
\begin{key}
  None
\end{key}
&
\begin{key}
  Cost of Textbooks
\end{key}
\\ \hline
Peanut Allergies &
\begin{key}
  Exp
\end{key}
&
\begin{key}
  Eat peanut protein or not
\end{key}
&
\begin{key}
  Allergic to peanuts at age 5
\end{key}
\\ \hline
Nonideal Weight &
\begin{key}
  Obs
\end{key}
&
\begin{key}
  Male/Female
\end{key}
&
\begin{key}
  Over/Under ideal weight
\end{key}
\\ \hline
Energy Drinks &
\begin{key}
  Exp
\end{key}
&
\begin{key}
  REDA/Control
\end{key}
&
\begin{key}
  change in RBANS
\end{key}
\\ \hline
Birth Weight &
\begin{key}
  Obs
\end{key}
&
\begin{key}
  Smoking/non
\end{key}
&
\begin{key}
  baby weight 
\end{key}
\\ \hline
Arsenic &
\begin{key}
  Obs
\end{key}
&
\begin{key}
 None
\end{key}
&
\begin{key}
  arsenic level 
\end{key}
\\ \hline
Attraction &
\begin{key}
  Obs
\end{key}
&
\begin{key}
 Age of interviewee
\end{key}
&
\begin{key}
 Most attract age in opp sex
\end{key}
\\ \hline
 \end{tabular}


 \begin{center}
{\bf Extensions }   
 \end{center}
% \item Sleep Deprivation Study
%   \begin{enumerate}
%   \item Think about how your analysis and conclusions might have
%     changed if you had subtracted the group means in the other
%     direction (sleep deprived mean -- unrestricted sleep mean).
% \begin{students}
%   \vspace{2cm}
% \end{students}    
% \begin{key}
%    {\it Our difference in means would be negative.}
% \end{key}

%   \item What parts of your analysis would have been the same, and what
%     parts (if any) would have turned out differently?  How would they
%     have been different (if at all)?
% \begin{students}
%   \vspace{2cm}
% \end{students}  

%   \item How would your conclusion about the study have changed (if at
%     all)? 
% \begin{students}
%   \vspace{2cm}
% \end{students}  

% \begin{key}
%    {\it The statistic would change signs and  the alternative
%     hypothesis would be $<$ instead of $>$, so we would look in the
%     left tail instead of the right to get our p--value.  P--value and
%     conclusions would not change.}
% \end{key}
%   \item  Investigate your predictions by making this change and
%     re-conducting your analysis. 
% \begin{students}
%   \vspace{2cm}
% \end{students}  




% \hspace*{-1cm} Investigate the effect that one observation can have on this analysis. 
% \item  Remove the improvement score of 45.6 from the unrestricted
%   sleep group, and re-conduct the analysis.  Comment on how much
%   impact this one observation has on your analysis and conclusion.
% \begin{students}
%   \vspace{2cm}
% \end{students}  

% \begin{key}
%   {\it Difference in means changes to 13.06 and the p-value is 0.018
%     after 4000 trials,
%     so evidence got slightly weaker.  However, we still make the
%     conclusion that we have strong evidence against the null hypothesis.}
% \end{key}
% \item Restore the 45.6 value but remove the -7.0 improvement score
%   from the unrestricted sleep group, and investigate the effect of
%   that change. 
% \begin{students}
%   \vspace{2cm}
% \end{students}  

% \begin{key}
%    {\it Difference in means changes to 18.09 and the p-value is 0.0015
%     after 4000 trials,
%     so evidence got much stronger.  However, we still make the
%     conclusion that we have strong evidence against the null hypothesis.}
% \end{key}
% \item  Notice that this research study involved slightly different
%   numbers of people in the two groups.  Suppose that you describe this
%   study to a friend, and he argues that the study is invalid because
%   of the unequal group sizes.  Describe how you would respond to your
%   friend, and be sure to include a description of how your analysis
%   took these unequal group sizes into account. 
% \begin{students}
%   \vspace{2cm}
% \end{students}  

% \begin{key}
%   {\it StatKey has no trouble drawing randomized groups of different
%     sizes, it just copies the shape of the original data.  The test is
%     valid for any sample sizes.  Having equal sample sizes is
%     preferable to get better power, but not necessary.}
% \end{key}
%   \end{enumerate}
    
\item Peanut Allergy Study
  \begin{enumerate}
  \item  Suppose the results of the experiment had been that 4 had
    become allergic in the peanut group (instead of 5) and only 36 had
    become allergic in the control group (instead of 35).  Explain how your
    approximate p-value would have been different in this case.  Also
    describe how the strength of evidence for the benefit of peanut protein
    would have changed.
\begin{students}
  \vspace{2cm}
\end{students}  

\begin{key}
  {\it  One fewer allergic kid in the treatment group and one more
     in control make this even stronger evidence against the null
     hypothesis. The difference in proportions becomes -0.125, and in 5000
     randomization trials, I never got one sample with this large a
     difference in sample proportions, so p--value is $< 1/5000 =
     .0002$  Our conclusion is the same.
  }
\end{key}

\item Suppose that all counts were divided by 5, so we had 1 allergy
  in the treatment group and 7 in the controls (out of 49 and 51
  kids).  Explain how your p-value would have been different in this
  case.  Also describe how the strength of evidence for the benefit of
  peanut protein would have changed.  
\begin{students}
  \vspace*{\fill}
  \newpage
\end{students}

\begin{key}
   {\it A good guess is that the same proportion in the smaller study
     provides weaker evidence. It does.  When I run 5000 trials with
     100 kids, the differences I got 16 values $< -0.117$ for a
     p-value of  $< .003$.}
\end{key}
  \end{enumerate}




  \begin{center}
  {\bf  More Examples}
  \end{center}

The following exercises are adapted from the CATALST curriculum at
\url{https://github.com/zief0002/Statistical-Thinking}. 


\item Teen Hearing Study

  Headlines in August of 2010 trumpeted the alarming news that nearly
  1 in 5 U.S. teens suffers from some degree of hearing loss, a much
  larger percentage than in 1988.\footnote{ Shargorodsky et. al.,
    2010. {\it Journal of the American Medical Association}}.  The
  findings were based on large-scale surveys done with randomly
  selected American teenagers from across the United States: 2928
  teens in 1988-1994 and 1771 teens in 2005-2006.  The researchers
  found that 14.9\% of the teens in the first sample (1988-1994) had
  some hearing loss, compared to 19.5\% of teens in the second
  (2005-2006) sample.
\begin{enumerate}
\item  Describe (in words) the research question. List the explanatory and
  the response variables in this study.   
 \begin{students}
  \vspace{2cm}
\end{students}

\begin{key}
  {\it Question:  Is the proportion of teens in the US with hearing
    loss still 14.9\%, or has it increased?\\
    Explanatory variable: year of survey\\
    Response: Some hearing loss.}
\end{key}
\item  Just as with the peanut protein therapy and sleep deprivation studies,
  this study made use of randomness in collecting the data. But the
  use of randomness was quite different in this study.  Discuss what
  type of conclusions can be made from each type of study and why you
  can make those conclusions for one study but not the other.  
 \begin{students}
  \vspace{2cm}
\end{students} 

\begin{key}
  {\it We can infer association back to the populations of teenagers
    (2004 and 1991), but it is not an experiment, so we cannot make
    causal inference}.
\end{key}

\item  Are the percentages reported above (14.9\% and 19.5\%)
  population values or sample values?  Explain. 
\begin{students}
  \vspace{2cm}
\end{students} 

\begin{key}
  {\it Sample proportions.  We cannot take a census to find the true
    population proportions.}
\end{key}
\item  Write out the null model for this analysis. 
\begin{students}
  \vspace{4cm}
\end{students} 
 
\begin{key}
  
\end{key}
\end{enumerate}


\begin{students}
\newpage
\end{students}

\item Mammography Study

A mammogram is an X-ray of the breast.  Diagnostic mammograms are used
to check for breast cancer after a lump or other sign or symptom of
the disease has been found.  In addition, routine screening is
recommended for women between the ages of 50 and 74, but controversy
exists regarding the benefits of beginning mammography screening at
age 40. The reason for this controversy stems from the large number
of false positives.  Data consistent with mammography screening yields
the following table:\footnote{{\it Annals of Internal Medicine}
  November 2009;151:738-747} 


\begin{tabular}{|l|c|c|c|}\hline
   & \multicolumn{2}{|c|}{Mammogram Results:}& \\
Truth: & Positive & Negative & Total\\\hline
Cancer & 70 & 90 & 160\\ \hline
No Cancer& 700& 9140&9840\\ \hline\hline
Total& 770 & 9230 & 10000\\ \hline
\end{tabular}

\begin{enumerate}
   \item   What percent of women in this study have breast cancer?
\begin{students}
  \vspace{\fill}
\end{students} 

\begin{key}
  $160/10000 = .016 = 1.6\%$
\end{key}
   \item   If the null hypothesis is that a woman is cancer free, what
     would an erroneous test result be?  Is that a false positive or a false
     negative? 
\begin{students}
  \vspace{\fill}
\end{students} 

\begin{key}
  {\it Being told she has cancer}
\end{key}
   \item  Estimate that error rate using these data. 
\begin{students}
  \vspace{\fill}
\end{students}

\begin{key}
  {  $700/9840 = 0.071 = 7.1\%$}
\end{key}
   \item  If a woman really has cancer, what would an error in the
     test be saying? Is that a false positive or a false
     negative? 
\begin{students}
  \vspace{\fill}
\end{students} 

\begin{key}
    {\it That she has no cancer, a false negative.}
\end{key}
   \item  Estimate that error rate using these data. 
\begin{students}
  \vspace{\fill}
\end{students} 

\begin{key}
{$90/160 =  .563 = 56.3\%$ \it That seems poor! }
\end{key}

If a patient tests positive for breast cancer, the patient may
experience extreme anxiety and may have a biopsy of breast tissue for
additional testing.  If patients exhibit the symptoms of the disease
but tests negative for breast cancer, this may result in the patient
being treated for a different condition. Untreated cancer can lead to
the tumor continuing to grow or spread. 
\item  Given the consequence of a false test result, is the
  false negative or false positive a larger problem in this case? Explain. 
\begin{students}
  \vspace{\fill}
\end{students} 

\begin{key}
  {\it I rate death from a cancer which should have been detected as
    more critical than the anxiety of a false positive, so I think
    false negatives are more important.}
\end{key}

\end{enumerate}

\begin{students}
\newpage
\end{students}

\item Blood Pressure Study
 
In a 2001 study, volunteers with high blood pressure were randomly
assigned to one of two groups. In the first group -- the talking group
-- subjects were asked questions about their medical history in the
minutes before their blood pressure was measured.  In the second group
-- the counting group -- subjects were asked to count aloud from 1 to
100 four times before their blood pressure was measured.  The data
presented here are the diastolic blood pressure (in mm Hg) for the two
groups.  The sample average diastolic blood pressure for the talking
group was 107.25 mm Hg and for the counting group was 104.625 mm Hg. 
            
\begin{tabular}{|l|c|c|c|c|c|c|c|c|}\hline
 Talking & 103& 109& 107& 110& 111& 106& 112& 100\\ \hline
 Counting& 98& 108& 108& 101& 109& 106& 102& 105\\ \hline
\end{tabular}

\begin{enumerate}
     \item  Do the data in this study come from a randomized
       experiment or an observational study?  Explain. 
\begin{students}
  \vspace*{\fill}
\end{students} 

\begin{key}
  {\it Randomized experiment because the treatment (talk or count) was
    assigned randomly.}
\end{key}
     \item  Calculate the difference in the means. 
\begin{students}
  \vspace*{\fill}
\end{students} 

\begin{key}
  {\it 2.625}
\end{key}
     \item  Write out the null model for this study.
\begin{students}
  \vspace*{\fill}
\end{students} 

\begin{key}
  {\it Mean blood pressure is the same for people talking or counting.}
\end{key}
     \item  Use our web app to do the appropriate test to determine if a
       difference this large could reasonably occur just by chance.
       Comment on the strength of evidence against
       the null model.
\begin{students}
  \vspace*{\fill}
\end{students} 


\begin{key}
  {\it Running 5000 trials of a randomization test, I got a p--value of
    .101 which gives only weak evidence against the null hypothesis of
    equal means.}
\end{key}
     \end{enumerate}
     
\begin{students}
\newpage
\end{students}

%    \item Social Fibbing Study

% A student investigated ``social fibbing'' (the tendency of subjects to
% give responses that they think the interviewer wants to hear) by
% asking students ``Would you favor a policy to eliminate smoking from
% all buildings on campus?''  She randomly assigned half the subjects to
% be questioned by an interviewer smoking a cigarette and the other half
% were interviewed by the same student but not while she was smoking.
% The results are displayed in the following table. 

% \begin{tabular}{|l|c|c|c|}\hline
%  & Favor Ban & Not Favoring Ban & Total\\\hline
%  Smoking & 43 & 57 & 100\\ \hline
% Not smoking& 79& 21&100\\ \hline\hline
% Total& 122 & 78 & 200\\ \hline
% \end{tabular}

%  Does the behavior of an interviewer affect the responses of the
%  people being surveyed? Use StatKey to do the appropriate test to
%  determine if a difference this large could reasonably occur just by
%  chance.  Comment on whether the difference in the percents provides
%  strong evidence  against the null model. \\
% \begin{key}
%   {\it Running 5000 trials of a randomization test, I got a p--value of
%     less than .002 which gives very strong evidence against the null
%     hypothesis of equal means.}
% \end{key}

\item Investigators at the UNC Dental School followed the growth of 11
  girls from age 8 until age 14.  Every two years they measured a
  particular distance in the mouth via xray ((in mm) .  Assume that
  they want to test ``Is the rate of growth zero?''.  The data are
  preloaded as ``Dental'' under \fbox{Two Quant}.  Note: ages are fixed by
  design, not randomly assigned.
     \begin{enumerate}
     \item Find the estimated least squares line.  Note: be sure that
       ``age'' is the explanatory variable in your plot. You may need
       to click \fbox{Swap Variables (X goes to Y)} to get that ordering.
\begin{students}
  \vspace{1.5cm}
\end{students} 

\begin{key}
  {\it $\widehat{\mbox{distance}} = 17.37 + 0.4795 x \mbox{age}$}
\end{key}

     \item How fast is this measurement changing?
\begin{students}
  \vspace{1.5cm}
\end{students} 

\begin{key}
  {\it It increases by an estimated .48 mm for each year.}
\end{key}

     \item What hypotheses are we testing?

  $H_o:$
\begin{students}
 \vspace{1cm}      
\end{students}
\begin{key}
  {\it  $\beta_1 = 0$   }
\end{key}

  $H_a$
\begin{students}
 \vspace{1cm}      
\end{students}
\begin{key}
  {\it  $\beta_1 > 0$   We don't expect a negative relationship.}
\end{key}

     \item Compute the p-value for the hypothesis test.
\begin{students}
  \vspace{1.5cm}
\end{students} 

\begin{key}
  {\it 0.0009  or 9 in 10,000 for me}
\end{key}

     \item Give the scope of inference.
\begin{students}
  \vspace{1.5cm}
\end{students} 

\begin{key}
  {\it Association in the sample.}
\end{key}

     \end{enumerate}


\end{enumerate}
