\def\theTopic{Readings 2}

\begin{center}
{\bf {\large Population and Sample}}
\end{center}

The science of statistics involves using a {\bf sample} to learn about
a {\bf population}.

{\bf Population}: all the units (people, stores, animals, ...) of
interest.

{\bf Sample}:  a subset of the population which gets measured or
observed in our study.

{\bf Statistical  Inference}: making a statement about a
{\bf population parameter} based on a {\bf  sample statistic}.

{\bf Parameter}:  a number which describes a characteristic of the
population. These values are never completely known except for small
populations which can be enumerated. We will use:\\
  $\mu$ to represent the population mean.\\
  $\sigma$ to represent the population's standard deviation
  (spread).\\
  $p$  to represent a population proportion.\\
  $\rho$ (the Greek letter ``rho'') for correlation between two
  quantitative variables in a population.\\
  $\beta_1$ slope of a true linear relationship between  two
  quantitative variables in a population.

{\bf Statistic}:  a number which describes a characteristic of the
sample and can be computed from the sample. We will use:\\
  $\xb$ to represent the sample mean (or average value).\\
  $s$  to represent the sample's standard deviation
  (spread).\\
  $\phat$  to represent a sample proportion.  (We often use a hat to
  represent a statistic.)\\
  $r$  for correlation between two  quantitative variables in a sample.\\
  $\widehat{\beta}_1$ slope of the ``best fitting'' line  between  two
  quantitative variables in a sample.

In this first unit, we will focus on parameter $p$ using sample
statistic $\phat$ to estimate it.

\begin{center}
  {\bf Representative Samples}
\end{center}

Because we want the sample to provide information about the
population, it's very important that the sample be {\bf
  representative} of the population. \\
In other words: we want the statistic we get from our sample to be
{\bf unbiased}.\\

\begin{center}
  {\bf Sampling problems}:\vspace{-.5cm}
\end{center}
\begin{list}{}{}
\item [\bf Convenience Sample]  is made up of units which are easy to
  measure. For example, to assess people's opinions on federal college
  loan programs, we interview students on a university campus.  Or to
  assess the presence of noxious weeds in the state, we select only plots
  of ground which are within 100m of a secondary highway. 
\item [\bf Non-response bias:] If people refuse to answer questions
  for a phone survey, or do not return a mailed survey, we have a
  ``non-response.'' When will non-responses cause bias in the results?
\end{list}

\begin{center}
  {\bf Ideal Samples}
\end{center}
 Ideally we will have a list of all units in the population and can
 select units {\bf at random} to be in our sample.\\ 
Random selection assures us that the sample will generally be
representative of the population.\\
 A {\bf simple random sample} is selected so that every sample of size
 $n$ has the same chance of being selected.  You can think of this as
 pulling names out of a hat (although it's better to use the computer
 to select samples since names in the hat might not be well mixed). 

  Simple random sampling is not the only way to get a random
  sample, and more complex schemes are possible.  If you run into a
  situation in which the population is divided into strata (for
  example university students live either on campus, in Greek houses,
  or non-Greek off campus housing, and you want to sample from each)
  you can use a stratified sample.  We will only use simple random
  sampling (SRS) in this course, and suggest that you consult a
  statistician or take more statistics classes if you need more
  complexity. 

 Non-response bias can be addressed with more work.  We would have to
 do a second (or further) attempt to contact the non-responders, then 
 check to see if they differ (in some important way) from those who
 responded the first time.  Again, this is a situation in which you
 would need further statistical expertise.

 Bias can also result from the wording of a poll, so writing questions
 is a science in its own right.  People tend to try
 to please an interviewer, so they might, for example, soften their
 attitudes toward breathing second-hand smoke if they know the interviewer
 smokes. 


