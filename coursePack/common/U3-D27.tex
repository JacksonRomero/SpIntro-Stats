\def\theTopic{Unit 3 Wrapup }
\def\dayNum{27 }

\begin{center}
{\bf {\large Unit 3 Wrapup }}\\
  {\bf Vocabulary} \vspace{-.3cm}
\end{center}
We have used these methods: \vspace{-.3cm}
\begin{itemize}
\item  Z-test for single proportion
\item  Z-based CI for one  proportion
\item  Z-test for difference in two  proportions
\item  Z-based CI for difference in two  proportions
\item  t-test for a single mean
\item  t-based CI for a single mean
\item  t-test for difference in two means
\item  t-based CI for a difference in two means
\item  t-test for mean difference in paired data
\item  t-based CI for mean difference in paired data
\end{itemize}
Know when to use $z$ versus $t$, and know what degrees of
freedom to use with the $t$'s.\\
Know the assumptions needed to use these methods.  

Important ideas:
\begin{itemize}
\item With random events we cannot predict a single outcome, but 
  we can know patterns we see with many repetitions.\vspace{.2cm}
\item  Interpretation of  confidence intervals -- they work in the
  ``long run''.\vspace{.2cm}
\item  Strength of evidence in Hypothesis Tests\vspace{.2cm}
\item  Assumptions to use z procedures on proportions\vspace{.2cm}
\item  Assumptions to use $t$ procedures on means \vspace{.2cm}
\end{itemize}
\begin{center}
  {\bf Practice Problems}
\end{center}
\begin{enumerate}
  \item What does the Law of Large Numbers (LLN) tell us about the
    distribution of the sample mean $\xb$ as $n$ gets big?\\
\begin{key}
 {\it  It gets tighter and tighter around the true mean, $\mu$.}      
\end{key}
  \item What does the Central Limit Theorem (CLT) tell us about the
    distribution of the sample mean $\xb$ as $n$ gets big?\\

\begin{key}
 {\it  It becomes normally distributed.  }      
\end{key}
   \item Which of  LLN and the CLT apply to sample
    proportions as $n$ gets big? \\
\begin{key}
 {\it Both, $\phat$ is just a mean of zero = failure and 1 = success
   random variates. }      
\end{key}
   \item How do we determine what to use as degrees of freedom for a
    $t$ distribution when we are conducting a one-sample $t$ test?\\
    \\ a two-sample t-test?\\

\begin{key}
 {\it 1 sample: $n-1$, 2 sample: $\min(n_1, n_2) -1$. }      
\end{key}

The following exercises are adapted from the CATALST curriculum at
\url{https://github.com/zief0002/Statistical-Thinking}. 

   \item Rating Chain Restaurants\footnote{ Rossman, A. J., Chance,
      B. L., \& Lock, R. H., (2009). {\it Workshop Statistics:
        Discovery with Data and Fathom}  (3rd ed.). Emeryville, CA:
      Key College Publishing. } 

    The July 2006 issue of Consumer Reports included ratings of 103
    chain restaurants. The ratings were based on surveys that Consumer
    Reports readers sent in after eating at one of the
    restaurants. The article said, ``The survey is based on 148,599
    visits to full-service restaurant chains between April 2004 and
    April 2005, and reflects the experiences of our readers, not
    necessarily those of the general population.''

    \begin{enumerate}
    \item  Do you think that the sample here was chosen randomly
        from the population of Consumer Report readers? Explain.
\begin{students}
       \vspace{2cm}
\end{students}
\begin{key}
 {\it No, it relies on voluntary response. }      
\end{key}
      \item  Why do the authors of the article make this disclaimer
        about not necessarily representing the general population?
\begin{students}
          \vspace{2cm}
\end{students}
\begin{key}
 {\it Because Consumer Reports readers are different in some ways --
   perhaps education level, perhaps, from the general population.}
\end{key}


      \item  To what group of people would you feel comfortable
        generalizing the results of this study? Explain. 
\begin{students}
          \vspace{2cm}
\end{students}
\begin{key}
 {\it Readers of Consumer Reports with enough time to answer the
   survey.  }      
\end{key} 
\end{enumerate}
 \item Emotional Support\footnote{Hite, S. (1976). {\it The Hite
       Report: A nationwide survey of female sexuality}. London:
     Bloomsbury.} 

   Shere Hite undertook a study of women’s attitudes toward
   relationships, love, and sex by distributing 100,000 questionnaires
   in women’s groups. Of the 4500 women who returned the
   questionnaires, 96\% said that they gave more emotional support
   than they received from their husbands or boyfriends.
   \begin{enumerate}
     \item  Comment on whether Hite's sampling method is likely to be
       biased in a particular direction. Specifically, do you think
       that the 96\% figure overestimates or underestimates the
       proportion who give more support in the population of all
       American women?  (In a voluntary response survey, those who do
       respond tend to have stronger and generally more negative
       opinions.) 
\begin{students}
       \    \vspace*{2.5cm}\\  \ 
\end{students}
\begin{key}
 {\it Again, it relies on voluntary response so it's likely to be
   biased against men. }      
\end{key}
     \item ABC News/Washington Post poll surveyed a random sample of
       767 women, finding that 44\% claimed to give more emotional
       support than they received.   Which poll’s result do you think
       are more representative of the population of all American
       women? Explain. 
\begin{students}
          \vspace{2cm}
\end{students}
\begin{key}
 {\it The ABC poll because it used random selection.}
\end{key}
     \end{enumerate}

     
   \item Balsa Wood

     Student researchers investigated whether balsa wood is less
     elastic after it has been immersed in water.  They took 44 pieces
     of balsa wood and randomly assigned half to be immersed in water
     and the other half not to be immersed in water.  They measured
     the elasticity by seeing how far (in inches) the piece of wood
     would project a dime into the air. Use the data file located on
     D2L.
     \begin{enumerate}

     \item  Before opening the data file, which applet should be used
       to create an interval estimate for the difference in elasticity
       (single proportion, single mean, two-sample proportion,
       two-sample mean)?  Explain. 
\begin{students}
          \vspace{2cm}
\end{students}
\begin{key}
 {\it StatKey two-sample means }      
\end{key}

     \item  The observed difference in mean elasticity between the two
       groups is 4.16 inches.  Explain why a confidence interval is a
       better summary of the data than just this difference in sample
       means. 
\begin{students}
\vspace{2cm}
\end{students}
\begin{key}
 {\it The point estimate does not include any information about how
   good an estimate it is.  The confidence interval has a length, so
   we can see that it is not exact, but has built in variability. }
\end{key}
     \item  Produce a 95\% bootstrap interval for estimating the
       actual size of the treatment effect of immersing balsa wood in
       water.  Describe the process by which you produce this
       interval, and also interpret what the interval means in the
       context of this study. 
\begin{students}
       \newpage
\end{students}
\begin{key}
 {\it  }      
\end{key}
     \end{enumerate}
   


   \item MicroSort$^{\textregistered}$ Study

       The Genetics and IVF Institute is currently studying methods to
       change the odds of having a girl or
       boy2. MicroSort$^{\textregistered}$ is a method used to sort sperm
       with X- and Y-chromosomes. The method is currently going
       through clinical trials. Women who plan to get pregnant and
       prefer to have a girl can go through a process called
       X-Sort$^{\textregistered}$. As of 2008, 945 have participated and
       879 have given birth to girls\footnote{Genetics \& IVF
         Institute, (2011). MicroSort. Genetics \& IVF
         Institute. Retrieved from \url{http://www.microsort.net/}. }
       \begin{enumerate}
       \item  Compute a 95\% interval to estimate the percentage of
         girl births for women that undergo X-Sort$^{\textregistered}$
         using a bootstrap method. 
\begin{students}
          \vspace{2cm}
\end{students}
\begin{key}
 {\it  }      
\end{key} 

       \item  Interpret your interval. 
\begin{students}
          \vspace{4cm}
\end{students}
\begin{key}
 {\it  }      
\end{key}

       \item Compute a 95\% interval estimate using a theoretical
         distribution.
         \begin{enumerate}
           \item Should you use a $z$ or a $t$ multiplier? 
\begin{students}
          \vspace{1cm}
\end{students}
\begin{key}
 {\it  }      
\end{key}
           \item What is the multiplier that should be used? 
\begin{students}
          \vspace{1cm}
\end{students}
\begin{key}
 {\it  }      
\end{key}
           \item What is the standard error of the sample proportion?
\begin{students}
          \vspace{1cm}
\end{students}
\begin{key}
 {\it  }      
\end{key}
           \item What is the margin of error of the interval estimate? \vspace{1cm}
           \item Give the interval estimate.
\begin{students}
          \vspace*{4cm}
\end{students}
\begin{key}
 {\it  }      
\end{key}
          
           \item Compare the interval estimate using the theoretical
             distribution to the interval estimate using Bootstrap
             method.  Are they similar in center?  Width? 
\begin{students}
          \vspace*{2cm}
\end{students}
\begin{key}
 {\it  }      
\end{key}
           \end{enumerate}

         \item  Suppose more data has been collected since 2008. If
           the number of women had increased to 3000 but the observed
           percent of girls stayed the same, what would you expect to
           happen to your interval? 
\begin{students}
          \vspace{2cm}
\end{students}
\begin{key}
 {\it It should get narrower, because SE will get smaller as $n$ gets bigger. }      
\end{key}

         \item  Test out your conjecture by creating a new interval
           using a sample size of 3000. Report your new interval
           estimate. Was your expectation in question 13 correct?

\begin{students}
          \vspace{2cm}
\end{students}
\begin{key}
 {\it StatKey two-sample means }      
\end{key} 

         \item  How many resamples (trials) did you run in your bootstrap
           simulation? 
\begin{students}
          \vspace{3cm}
\end{students}
\begin{key}
 {\it  }      
\end{key}

         \item  What is the difference between sample size and number
           of trials? 
\begin{students}
        \vspace{4cm}
\end{students}
\begin{key}
 {\it  }      
\end{key}

         \end{enumerate}
         
       \item 
         Anorexia nervosa is an eating disorder characterized by
         the irrational fear of gaining weight and a distorted body
         self-perception.  Several 
         psycho-therapy methods have been tested as  ways to treat
         individuals suffering from anorexia.  The data set available
         on D2L gives the results of a study using cognitive
         behavioral therapy (CBT) and the pre-and post-treatment
         weights of 29 patients, along with the change in weight. 
         You can get the descriptive statistics you
           need from the Rossman-Chance web app.   
         \begin{enumerate}
         \item\label{int1}  Build a 99\% confidence interval for the
           difference in true means.   
            Should we use the single or two-sample mean
           formula for SE?  Explain, then give the confidence
           interval. 
\begin{students}
          \vspace{2cm}
\end{students}
\begin{key}
 {\it  }      
\end{key}


         \item Would a 95\% confidence interval be wider or narrower?
           Explain. 
\begin{students}
          \vspace{2cm}
\end{students}
\begin{key}
 {\it Narrower. It has less probability in the center, so that moves
   the cutoffs in closer to 0. }      
\end{key}

         \item  Based on the confidence intervals, do the patients
           seem to improve (where improvement is based on increasing
           weight)?  What significance level is being used?
\begin{students}
          \vspace{2cm}
\end{students}
\begin{key}
 {\it  }      
\end{key}
           

         \item  Using the theoretical distributions, do a hypothesis
           test to answer this question.  Write the null and
           alternative hypotheses, calculate the standard error and
           the test statistic, and find the p-value.  Does the
           hypothesis test agree with the results for the confidence
           interval?  Explain. 
\begin{students}
          \vspace{4cm}\newpage
\end{students}
\begin{key}
 {\it  }      
\end{key}

         \end{enumerate}
       \item As mentioned in the last class, sports related
         concussions are a big concern.    One study
         investigated whether concussions are more common among male
         or female soccer players.  The study took a random sample of
         college soccer players from 1997 – 1999.  Of 75,082 exposures
         (practices or games) for female soccer players, 158 resulted
         in a concussion while 75,734 exposures for men resulted in
         101 concussions.  Does this show a gender difference in
         concussions rates among collegiate soccer players?  
         \begin{enumerate}
         \item  Write the null and alternative hypotheses. 
\begin{students}
          \vspace{2cm}
\end{students}
\begin{key}
 {\it  }      
\end{key}

         \item  We would like to use a theoretical distribution to
           conduct this  test. 
          \begin{enumerate}
           \item List the four assumptions for a hypothesis test
             (Activity 20).  Is  each one met?  
\begin{students}
          \vspace{3cm}
\end{students}
\begin{key}
 {\it  }      
\end{key}
           \item  Calculate the standard error of the difference in
             sample proportions. 
\begin{students}
          \vspace{1cm}
\end{students}
\begin{key}
 {\it  }      
\end{key}
           \item  Calculate the z test statistic.
\begin{students}
          \vspace{1.52cm}
\end{students}
\begin{key}
 {\it  }      
\end{key}

           \item  Find the p-value.
\begin{students}
          \vspace{1cm}
\end{students}
\begin{key}
 {\it  }      
\end{key}

           \item  Write-up your results using all five components
             required.  Use a 10\% significance level to make your
             decision.  
\begin{students}
          \vspace{2cm}
\end{students}
\begin{key}
 {\it  }      
\end{key}
\end{enumerate}

         \item  Instead of performing a two-proportion z-test, create
           a confidence interval to estimate the difference
           in the concussion rates between male and female soccer
           players.
           \begin{enumerate}
           \item  What confidence level should be used? 
\begin{students}
          \vspace{1cm}
\end{students}
\begin{key}
 {\it 90\% }      
\end{key}
           \item  Give and interpret the interval.
\begin{students}
          \vspace{3cm}
\end{students}
\begin{key}
 {\it  }      
\end{key}

           \item  Does the interval agree with your conclusion from
             the hypothesis test?  Explain. 
\begin{students}
          \vspace{2cm}
\end{students}
\begin{key}
 {\it  }      
\end{key}

\item Write-up the results of your confidence interval (including
  method 
  used (and number of trials if appropriate), interval estimate,
  interpretation of the interval estimate, and conclusion regarding
  the null hypothesis).
\begin{students}
          \vspace{2cm}
\end{students}
\begin{key}
 {\it  }      
\end{key}

    \end{enumerate}

  \end{enumerate}

\end{enumerate}


